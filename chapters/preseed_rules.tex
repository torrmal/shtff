\chapter{The Three Rules of Pree-Seed Stage}

Picture yourself in a dimly lit room where reality itself seems to flicker and waver, like a poorly tuned TV channel picking up signals from another dimension. This is the Idea-stage of a startup, where your idea exists in a once again quantum superposition of brilliant and catastrophic, alive and dead, unicorn and paperweight. Like Schrödinger's startup, until someone opens the box of market validation, you exist in all states simultaneously.

In this realm of pure potential, your idea floats like a ghostly butterfly in the void, neither fully formed nor completely imaginary. It's less a business plan and more a shared hallucination between you and whoever's brave enough to look into your eyes when you speak about it. This is where the Three Rules of Idea-Stage Reality emerge, they exist to keep you from dissolving into the quantum foam of possibility before you've even learned to manipulate it. (I made them up, I didn't so much invent them as discover them, like ancient runes carved into the walls of a startup incubator bathroom):

\setcounter{section}{1}
\section*{Rule 1: All Startup Ideas Are Shit Ideas Until They Aren't}
\addcontentsline{toc}{section}{Rule 1: All Startup Ideas Are Shit Ideas Until They Aren't}

Sun-shine, Let's get real here, Investors: They're basically professional truffle pigs, rooting through the mud of mediocrity hoping to find that one precious nugget that doesn't smell like desperation and delusion. At this point, chances are, your idea is about as useful as an ejector seat in a helicopter. No one, other than your daddy is there to worship at your PowerPoint altar or marvel at your revolutionary plan to disrupt the artisanal sock puppet market, your groundbreaking vision it's just more static in the cosmic background radiation of bad startup ideas, right up there with  "blockchain-enabled toothbrush". Maybe..., most of times. 

Some other rare-times you got a solution so brilliant it'll make Einstein's hair look even crazier, but the problem? It's smaller than a New Yorker's concept of reasonable rent. That's when these venture capital types hit you with that "too early" wildcard, which is fancy-speak for "damn, you saw this coming before anyone else could spell 'disruption.', yet we dont know if we will see money while we are still kicking" And sure as Trump's spray tan isn't found in nature, some of these tiny-ass problems eventually blow up bigger than a Kardashian scandal and they will need a solution.

The real game here, Love, and trust me on this like you'd trust a New York rat's restaurant recommendations, because they have tried them all: is surviving long enough for the market to wake up and smell the disruption or for you to find a real problem. Sometimes you'll find yourself sitting pretty, with exactly the right solution when the world finally catches up to your genius. Other times? You'll have to take your precious baby of a solution out for a spartan cliff drop, because while the problem's still real as rent control arguments, your solution needs a do-over. But this time? This time you've got all those battle scars and war stories to guide you. Think of it as your startup's second draft, and baby, everyone knows the first draft is just where you spill words onto the page like a preschooler pouring sugar and salt at the same time on a drink they've made for their parents.

\subsection*{So how did we learn this rule?}

It's late 2022, the market is awful, there's a war in Ukraine that took the world by surprise, everyone's worried about World War III, we're post-COVID, and investor checks aren't just something you find in your cereal box anymore. Hell, even Sam Altman's mother didn't know about ChatGPT or what her son actually did for work then. I'm on a call with George Mathew from Insight Partners (you can Google him later, but think "really good investor at a really good firm"). Don't ask me how we connected, at that point we were basically the startup equivalent of those people who slide into everyone's DMs with "hey beautiful." But hey, we spammed the hell out of this town like proper founders should, and the proof is we somehow landed that call. Something about our pitch caught his eye, and there he was, methodically dissecting my presentation like a surgeon performing an orchiectomy, the kind where you're pretty sure your startup will never see the light of offspring again. Meanwhile, my soul slowly left my body and started browsing LinkedIn for consulting jobs, wondering if I was too old to become a surf instructor somewhere nice in Brazil.

Mercifully, we'd only scheduled 15 minutes. When that blessed 1/4 mark hit, I practically screamed "OH WOULD YOU LOOK AT THAT!!, LOOK AT THE TIME, GOTTA RUN!" like a teenager ditching prom with their dignity still somewhat intact. I had a very important meeting with my mirror scheduled, where I planned to stare deeply into my own eyes and whisper "what the actual fuck?"

The question haunting me was deliciously simple: What mystical unicorn were we seeing that others couldn't spot with a telescope? What neon-bright red flags were others seeing that we were blind to? Who was the crazy one here? In hindsight: My guess? Oh darling, we all were wearing different flavors of crazy pants in this circus. 

I still remember vividly the day I called my co-founder Adam Carrigan. Picture this: it's a Sunday afternoon, and I'm interrupting what sounds like one of those trendy online workout sessions. You know the type; where fit people on screens make you feel bad about your life choices while you wheeze like a broken accordion.

"Hey mate, wanna quit your cushy consulting gig?" I asked, with all the casual nonchalance of someone suggesting we grab coffee, not torpedo our careers. He responded: "Yes yesterday, but no, it depends, what's up?"

Through his labored breathing (apparently planking and important life decisions go great together), I could practically hear his brain short-circuiting. We both had great paying jobs, finally, the kind your parents brag about at family gatherings, after a year of: oh well, they work on their computer playing businessmen. Our previous startup adventure had left us with enough emotional scar tissue to make a therapist retire early. But here we were, like recovering adrenaline junkies eyeing a bungee cord, itching for another fix.

"I think databases will become enterprise AI, they will have a Mind" I declared, probably sounding like a fortune cookie having an existential crisis.

Now, Adam has the built-in risk aversion of a British man facing unscheduled tea time, we're talking about he is the friend who considers jaywalking a gateway crime to full-blown anarchy. But to my utter bewilderment, this gentleman didn't even hesitate. If my partner in business, with whom I'd previously achieved the impressive feat of turning perfectly good savings money into absolutely nothing during our last startup adventure, was ready to leap back into the circus of entrepreneurial chaos, then surely this idea must've been pure, unadulterated, 24-karat gold, right?

...Right?

After back and forths on perfectioning our thoughts. We had this wild conviction that Machine Learning would eventually evolve into applied AI, you know, AI that actually does useful stuff instead of just crushing board games, PhD papers, or writing suspiciously existential poetry about cats. We believed, with the fervor of shroomed prophets, that every company on Earth would eventually crave this magical decision-making juice like it was the last bottle of water in the desert. And the market size? For enterprise AGI? Oh darling, just every company in existence. No pressure. No biggie. Just casual world domination.

But here's the punch: this was pre-2023, when AI wasn't the corporate equivalent of avocado toast, something every self-respecting CEO had to have on their agenda. Our solution, while technically impressive (if I do say so myself), required more attention than a needy ex with abandonment issues. In retrospect, in many ways, it felt like architects trying to sell flying houses before humans had mastered the wheel. Technically brilliant, practically useless. So, at this time, after having put tears and sweat into it. For most people, still, our idea was most of the time shit or at least not gold.

A few weeks after my disastrous pitch to George, I found myself in NYC for a wedding. With the audacity of someone who's been rejected so many times they've achieved enlightened zero-fucks-given status (a state I hope you too will reach), I messaged George: "Remember that startup you verbally eviscerated? Want to grab coffee?" To my eternal surprise, he agreed to 15 minutes of coffee. At the time I did not know him, so I thought it was probably out of morbid curiosity, but; as I later learned that day: he's actually a really cool and curious guy. He just happens to have the bullshit filter of a nuclear-grade truth detector.

Until then, we'd been blessed with magical thinking investors, you know, the kind who see a couple of passionate nerds and think "Well, they'll either change the world or create a spectacular failure worth telling stories about." Later, I will write about how crucial those investors were for our business. But George? George was a product person. He saw the holes in our plan like a wine mom spots judgment at a PTA meeting.

In my notes from our first call, there was this gem: "You're crazy if you think you can compete with existing players that already have the data." We saw ourselves as market disruptors, thinking people would flock to our solution like teenagers to a TikTok trend. But convincing companies to switch data systems is like trying to convince someone to move houses just because you invented a slightly better toilet. Sure, some might bite, but most people are perfectly happy with their current crapper, thank you very much.

The way investors think is brutally simple: how likely is it that this company will become absolutely massive? The question George was probably asking himself was: "How likely is it that these adorable dorks will convince everyone to abandon their perfectly functioning data infrastructure and jump ship to their solution?" About as likely as finding a llama-corn (that rare hybrid born nine months after a llama had a magical night of passion with a unicorn) sipping a pumpkin spice latte at your local Starbucks. But then, while having my profound heart-to-heart with the mirror, it hit me like an epiphany: we didn't need to move anyone's data! We just needed to integrate with every single data source that existed. Forget being data hoarders, we'd become the master plumbers of the AI era, connecting all the pipes without stealing anyone's precious data-toilet.

As we met for coffee, I had 15 minutes and I explained the epiphany with the enthusiasm of a hippie that has discovered that they too can brew their own kombucha, all of a sudden I too saw the parts of the puzzle make sense, like when you go from staring at a Magic Eye poster for three hours seeing nothing but static to suddenly spotting that 3D dolphin that was hiding there all along, mocking your perception. The pieces started falling into place faster than Silicon Valley startups pivot to AI nowadays. It was like watching Bob Ross paint one of his "happy little accidents", starting with chaos and ending with something that actually made sense.

Before we knew it, we were walking the streets of New York way past those promised 15 minutes. And then George dropped the kind of validation bomb that hits different, the kind that makes you feel like maybe, just maybe, you're not completely delusional after all. "This idea," he said, with the gravity of someone who's seen enough startups to develop PTSD, "if you pull it off, this is how a giant business is built." Then came the classic investor one-two punch: "Now you just need to execute and validate." Which in startup speak translates to "You've got a tiger by the tail. Don't screw this up". All of a sudden a "shitty idea" started to look more like a good idea!

\setcounter{section}{2}
\section*{Rule 2: You need to grow the skin of a fucking lizard, you can't give up!}
\addcontentsline{toc}{section}{Rule 2: You need to grow the skin of a fucking lizard!}

So for many of us, this will be a painful journey, the sooner you realize this, the better. When it comes to early stage fundraising, If you haven't yet mastered the art of getting rejected more times than a Nigerian prince's email asking for social security numbers, don't sweat it. It's wasn't technically required in your application form as an entrepeneur. However, this line of work will give you a Post Doc in rejection faster than you can say "so,.. what do you think?." Your skin will grow thicker than a rhinoceros wearing kevlar, it has to. It's like asking if a boxer gets used to being punched in the face. Who knows? Maybe. Either way, You'll build up immunity to rejection like you will build up immunity to ramen; it's not that you enjoy it, you just lose all form of taste. 

\begin{quote}
    ``Jorge, what you need to understand is that in this game no one will believe in you, unless you are the one that believes the most'' \\
    \hfill --- Chetan Puttagunta, one of the many times I have felt like I am losing the battle.
\end{quote}


The important thing is to not give up, to keep learning, iterating. To see opportunities for survival where others don't. Our fable to this rule, is a near death experience. Six or so months into MindsDB. Chon Tang from Skydeck (you can research about him, but essentially like Professor X to the X-Men Academy for startups), our first investor and now friend. This guy had front-row seats to our fundraising circus: watching us pitch to everyone from venture capitalists to that one guy who looked suspiciously like a janitor but we still believed they could be our angel investor, maybe. We were like desperate singles at last call, hitting on everyone in sight.

But holy quantum algorithms, were we working harder than a hamster that found a bag of speed next to a bag of nuts! There's something absolutely intoxicating about building a product from nothing but the fumes of your hallucinations while simultaneously trying to figure out what the hell a "go-to-market strategy" is. Chon met with us for countless hours, probably questioning his choices that one moment he said, 'welcome to Skydeck', but sticking around anyway. He was what every entrepreneur needs at that stage: a "fuck it, let's do it" bouncing wall who'd listen to our fever dreams with the patience of a Buddhist monk at a heavy metal concert.

Because here's the thing about the early startup game: the more lost you are, the more time you can spend debating the stupidest of things, like number of pixels for the left margin of paragraph four on your website, than actually building something and trying it with the market. But with Chon, we were all about trying things like teenagers discovering cooking for the first time: throwing oil-drenched spaghetti at the wall with the enthusiasm of a blockchain newb at a Web3 conference, completely oblivious to the fact that all that oil meant nothing would ever stick. To the outside eye, we were basically conducting a masterclass in how NOT to make pasta while thinking we were the next Gordon Ramsay of startups, but from the inside we were working harder than ever to find that one spaghetti we did not put oil in..

The market wasn't ready for us, or maybe we were too busy being the proud parents of our "super cool solution" to notice that -businesses actually- need, you know, business fundamentals. Details, etc.

So Chon invites me to lunch, and I'm thinking "free food!" but no, it's intervention time. He's giving me the "you fought a really good fight" speech to someone that can't open their eyes from how swollen the wounds are, one more punch in the face it's not worth it, people already saw you being beaten to death, they just want you to be able to walk a few months later. He knows we tried harder than a vegan trying to convince Tyson Foods board about broccoli's exciting flavor profile.

There I am the next day, packing up our office into a sad little box (which mostly contained empty Red Bull cans), and just like that is when fate decides to play one of its practical jokes. Some other team, the darling of our cohort, cancels their investor meeting, because they can't say yes to everyone, and suddenly I'm the backup of the backup dancer called to center stage. 

I show up to meet this investor who's crankier than a cat in a bathtub about driving all the way to Berkeley from god knows Santa-somethin' for a cancelled meeting. He suggests a phone call instead so he can beat traffic across the bridge, and I'm like "Sure, one more bullet in the chamber!" I call and pitch with the confidence of someone who wasn't packing their failure into a cardboard box as they spoke, channeling everything opposite to that time Adam and I fumbled the bag in San Francisco way back then. Because now, I know it's not over until it's over.

The phone call ends with that classic investor line: "Aight, send me your terms and deck, I'll look at it while driving." Translation: "Cool story bro, I'll file this under 'Never in a Million Years.'" But remember Rule \#2? Never give up? Well, we sent that pitch deck faster than he could put his blinkers on, complete with a signed YC Compatible SAFE agreement (see the glossary for more details about SAFE agreements) and our bank wiring detailed instructions next to the "we will make you rich and looking forward to catching up again soon".

Fast forward a week or two, because time flies when you're busy writing startup obituaries, and my phone rings. This investor, sounding like someone who just discovered an untouched pizza in their fridge after a rave, asks if we've received the transfer they made days ago. Just like that, like a zombie startup that learned to tap dance, we moonwalked right out of the startup graveyard. And here we are, telling the tale like contestants who not only survived the Hunger Games but also managed to steal President Snow's Netflix password in the process.

Before you go full armadillo—rolling through startup land with a hide so thick you can eat rejection for brunch and crap out disappointment for dessert. You absolutely, unavoidably, must stare into the bathroom mirror and ask the startup equivalent of “Does this haircut make me look like I eat crayons?” Here it is: Can you imagine waking up for the next ten years, every single morning, and still giving a quantum-fluctuation of a damn about THIS problem? Not “Do I adore my brainchild solution today, the slick gadget I MacGyvered together after three Red Bulls and a spiritual awakening in my shower?” No. Sober up: If your solution is just a spicy fever dream and the actual problem bores you sideways, hit pause. Don't let your founder romance write checks your motivation can’t cash. So here is a subrule, or an exception to the rule of not giving up:

\subsection*{Rule 2.1: YOU NEED TO BE IN LOVE WITH THE PROBLEM!}

This is so painfully obvious that as I write it, it makes me want to staple my own eyelids shut, but here we are, having this conversation like two people trying to assemble furniture while high. It's one of the most important rules to follow if you want to survive this madness. And I don't mean that in the sense that you need to always be the kind of person who can look at a spreadsheet full of red numbers and think "challenge accepted" rather than "where's the nearest bridge?". I mean that if the problem, the industry that you picked to solve is not something you would want to solve for the next many years, don't do it. \textbf{Life is in fact short.}

\subsection*{Let me tell you about my face-plant that relates to this whole circus:}

Picture this: Adam and I, we were just about to crash and burn with our previous hardware startup that we failed at (Real Life Analytics). We were broker than a piggy bank at a hammer convention, questioning every life decision that led us to believe hardware was a good idea. (Pro tip: Hardware is like dating a supermodel, looks amazing, but the costs of that relationship can kill you). If you've succeeded in hardware, you are my hero, you can and you should look down on software people like they're playing with Fisher-Price toys while you're wrestling bears. It's not even apples and oranges; it's more like comparing one of those walkers that toddlers use to learn their first steps to a rocket.

So there we were, desperate enough to email random people on LinkedIn. By some miracle, this prime-cut hardware-focused fund invited us to San Francisco. I mean,... they didn't pay for our flights, hotels or anything, they just said, if you are in the Bay we can meet in our office, so clearly East coast to SF was a stone throw away, so yes, we were in the area, and we pitched our hearts out to a partner whose face had all the emotional range of Poker world champion. If we'd pitched to actual concrete, we might've gotten more emotions from it. They did ask good questions, and by then we had answers smoother than a VC's LinkedIn humble-brag about their latest unicorn, but our souls had already left the building.

Meeting over, Adam and I stood at that, some random pier where their office was, staring at the bay like two rejected contestants from The Bachelor. "So... this is the end, right?" Adam shared with a smirk, but also with all the enthusiasm of someone announcing their own funeral. "What shall we do?"

Our hearts were more broken than a developer's code after a coffee spill on their only backup. We knew that their "this is interesting you guys, we'll reconvene and message you tomorrow" really meant "Thanks for flying across the country you bosons, but we could've crushed your dreams over the phone."

And then we did what any self-respecting failed entrepreneurs would do: we decided to get absolutely hammered good-bye. Like college freshmen with their first fake IDs, we bought the cheapest booze we could find and went to town.

Next morning, we woke up on our friend's Icelandic ice breaker ship (because believe it or not that's a thing people have in San Francisco), splitting headaches and phones buzzing like angry wasps. I answer, and it's no other than the investors from yesterday: "Hey guys" -this time all emotions- "we'll be in the office if you want to discuss investment terms."

I jumped up faster than a cat in a field of mirrors, found Adam, and we sprinted to their office. Only just before we arrived, looking like we'd just lost a fight with a seafood buffet, smelling like if we had showered in lobster juice, did I realize we probably shouldn't have given up so early. There was still one bullet in the chamber, but we'd already written our own obituary yesterday with ink made out of tequila, we knew we could not see ourselves doing hardware for advertisemnt in the following years. The solution we loved, it was what two nerds like us would enjoy building for fun, but we were looking for a problem to our solution, we couldn't care less if people did or not see the right billboard ad on their way home.

I called back and declined the meeting, which probably made me wonder if we were the first founders in Silicon Valley history to turn down money because their hangover felt like their brain was being used as a drum set by a toddler that just discovered sugar. But here's the real lesson: we had already emotionally checked out. The day before, we'd basically done the startup equivalent of not just burning bridges, but nuking them from orbit, tap dancing on the ashes, and sending a postcard about it. We'd mentally resigned harder than someone who quits their job by storming into their boss's office with an opening line such as "I am glad your mother did not swallow you, but... ", proceeds skywriting with their middle finger "I QUIT"  and then parachuting into their exit interview wearing a banana costume. The desire to solve that particular problem was born more dead than my will to ever drink tequila or whatever that elixir was that we killed our last intentions in.

For years, I wondered what could've been if we had taken that last meeting. But now I know it was for the best. We needed that failure like a teenager needs embarrassing yearbook photos. It taught us the most valuable lesson: Before you start taking checks from people, before you burn your savings, your time, put at risk your sanity, your health, your relationships, your sleep, your everything. It's ok to stop, if you don't love the problem your startup is trying to solve, on the other hand, if you see yourself working on this for a very long time, never give up, because sometimes, just sometimes, your lowest moment is right before someone decides to bet on your crazy dreams. 


\setcounter{section}{3}
\section*{Rule 3: It is never, never too late to fuck it up!}
\addcontentsline{toc}{section}{Rule 3: It is never, never too late to fuck it up!}

Listen, I need you to tattoo this somewhere visible, maybe on your forearm, maybe on your co-founder's forehead: the startup gods are petty, vindictive little gremlins who feed exclusively on the tears of founders who dared to feel good about themselves for more than eleven consecutive seconds. The moment you let that champagne cork fly before the wire transfer clears? That's when the universe puts on its favorite pair of steel-toed boots and aims directly at your reproductive organs.

This doesn't mean you should never make mistakes: that's as impossible as finding a VC who doesn't mention their "thesis" within the first three minutes of meeting you. No, mistakes are the breakfast cereal of entrepreneurship; you're going to eat them whether you like it or not. But here's the thing: startups are a competition, fundraising is a blood sport dressed up in Patagonia vests, and like any competition where people, other foudners, your competitors pretend to be civilized while secretly wanting to watch you choke on your own pitch deck, you should never, ever, bring your guard down. Because the best of your competitors won't.

This rule is about strangling that treacherous little narrator in your skull, the one who sounds like a life coach who just discovered cocaine and Tony Robbins on the same weekend, whispering "You're a genius! The hard part is over! Maybe treat yourself to a Tesla!" the moment a term sheet or a SAFE agreement lands in your inbox. That voice is not your friend. That voice is a sleeper agent planted by the universe specifically to destroy you. That voice has ended more promising startups than lines did to Wall Street in the '80s, except at least those guys believed they had good parties on the way down.

If I am not being clear, let me say it again: The biggest mistake you can make is to let your ego give your ego a handjob. I've seen it happen. Hell, I've been the one with my pants down, metaphorically speaking, spiritually exposed, thinking I was about to close a round when in reality I was about to close my laptop and cry into a bowl of cereal. Everything great; and I mean absolutely, spectacularly, against-all-odds great! that has happened at MindsDB traces back to this rule: we stayed more paranoid than a drug dealer at a family reunion, more vigilant than a parent at a playground full of suspiciously friendly adults. It's especially and more so, right at the exact moment we thought we'd accomplished something.

Because that's when you're most vulnerable. When you're failing, you're sharp. You're hungry. You're a feral raccoon digging through the dumpster of opportunity with the focus of a Navy SEAL who hasn't eaten in days. But the second you think you've "made it"? You become a golden retriever who found a sunny spot on the carpet—happy, stupid, and completely unaware that someone's about to vacuum your entire existence.

When it comes to fundraising specifically, let me be grotesquely clear: until the money is sitting in your bank account, visible, countable, not "pending," not "processing," not "the wire should hit by EOD," you have accomplished exactly nothing. You have done precisely zero shit. You are standing at the altar in your wedding dress, and the groom could still be in a taxi heading to the airport with your maid of honor. Until you can log into that bank account and see those beautiful digits staring back at you like a lover who actually showed up, you need to operate under the assumption that everything—and I mean everything—is actively conspiring to implode.

The final stages of funding are particularly sadistic. Fewer variables, sure, but each one is a landmine of details, often times wrapped in legal jargon and "just one more thing" emails from lawyers billing \$900 an hour to ask questions that you are fooled to blieve a moderately intelligent goldfish could answer, and thus you can delegate. Hear the word micro-management? This is your time to do it, if managing at all, try not to delegate, and if you do, check everything, make sure you are on top of it all. One wrong signature, one weird clause, one partner at the fund who had a bad oyster at lunch and suddenly decides they need to "revisit the terms," and you can watch months of work disintegrate in front of you faster than a sandcastle at high tide, faster than my dignity at that hardware pitch in San Francisco.

Forget what anyone tells you otherwise. Forget the "we're basically done" and the "it's just paperwork now" and the "this is a formality." Those phrases are the startup equivalent of "I promise I'll pull out" or "the check is in the mail" or "I'm fine" when your partner says it with that specific tone. They are lies. Beautiful, hopeful, relationship-ending lies. And it's when your most vigilant you should shine.

The undeniable truth compadre, is that no one gives a flying fuck if you almost won something. Your landlord doesn't accept "almost." Your employees can't pay rent with "we were so close." Your next employer if you have to get back to that, is never going to say "Oh, you ALMOST raised a ton of money? Well then, let's just skip the rest of this interview, you are hired." Almost is the participation trophy of the business world, and participation trophies are just shiny reminders that you showed up and still managed to lose because you got distracted when the ball was coming straight to your hand.

So celebrate nothing before that moment of truth. Actually, fudnraising is just a milestone, out many ahead, celebrate your first paying customer more than your first round of funding. And for the love of whatever deity you pray to when your runway drops below three months, keep your guard up until you can physically see the zeros in your account. Then, and only then, can you allow yourself exactly one (1) fist pump in the privacy of your bathroom before immediately panicking about whether you can actually deliver on all those promises you made to get that money.

Welcome to the pre-seed stage. Try not to die.