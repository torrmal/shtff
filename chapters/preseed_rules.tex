\chapter{The Five Rules of Pree-Seed Stage}

Picture yourself in a dimly lit room where reality itself seems to flicker and waver, like a poorly tuned TV channel picking up signals from another dimension. This is the Idea-stage of a startup, where your idea exists in a once again quantum superposition of brilliant and catastrophic, alive and dead, unicorn and paperweight. Like Schrödinger's startup, until someone opens the box of market validation, you exist in all states simultaneously.

In this realm of pure potential, your idea floats like a ghostly butterfly in the void, neither fully formed nor completely imaginary. It's less a business plan and more a shared hallucination between you and whoever's brave enough to look into your eyes when you speak about it. This is where the Five Rules of Idea-Stage Reality emerge, they exist to keep you from dissolving into the quantum foam of possibility before you've even learned to manipulate it. (I made them up, I didn't so much invent them as discover them, like ancient runes carved into the walls of a startup incubator bathroom):

\setcounter{section}{1}
\section*{Rule 1: All Startup Ideas Are Shit Ideas Until They Aren't}
\addcontentsline{toc}{section}{Rule 1: All Startup Ideas Are Shit Ideas Until They Aren't}

Sun-shine, Let's get real here, Investors: They're basically professional truffle pigs, rooting through the mud of mediocrity hoping to find that one precious nugget that doesn't smell like desperation and delusion. At this point, chances are, your idea is about as useful as an ejector seat in a helicopter. No one, other than your daddy is there to worship at your PowerPoint altar or marvel at your revolutionary plan to disrupt the artisanal sock puppet market, your groundbreaking vision it's just more static in the cosmic background radiation of bad startup ideas, right up there with  "blockchain-enabled toothbrush". Maybe..., most of times. 

Some other rare-times you got a solution so brilliant it'll make Einstein's hair look even crazier, but the problem? It's smaller than a New Yorker's concept of reasonable rent. That's when these venture capital types hit you with that "too early" wildcard, which is fancy-speak for "damn, you saw this coming before anyone else could spell 'disruption.', yet we dont know if we will see money while we are still kicking" And sure as Trump's spray tan isn't found in nature, some of these tiny-ass problems eventually blow up bigger than a Kardashian scandal and they will need a solution.

The real game here, Love, and trust me on this like you'd trust a New York rat's restaurant recommendations, because they have tried them all: is surviving long enough for the market to wake up and smell the disruption or for you to find a real problem. Sometimes you'll find yourself sitting pretty, with exactly the right solution when the world finally catches up to your genius. Other times? You'll have to take your precious baby of a solution out for a spartan cliff drop, because while the problem's still real as rent control arguments, your solution needs a do-over. But this time? This time you've got all those battle scars and war stories to guide you. Think of it as your startup's second draft, and baby, everyone knows the first draft is just where you spill words onto the page like a preschooler pouring sugar and salt at the same time on a drink they've made for their parents.

\subsection*{So how did we learn this rule?}

It's late 2022, the market is awful, there's a war in Ukraine that took the world by surprise, everyone's worried about World War III, we're post-COVID, and investor checks aren't just something you find in your cereal box anymore. Hell, even Sam Altman's mother didn't know about ChatGPT or what her son actually did for work then. I'm on a call with George Mathew from Insight Partners (you can Google him later, but think "really good investor at a really good firm"). Don't ask me how we connected, at that point we were basically the startup equivalent of those people who slide into everyone's DMs with "hey beautiful." But hey, we spammed the hell out of this town like proper founders should, and the proof is we somehow landed that call. Something about our pitch caught his eye, and there he was, methodically dissecting my presentation like a surgeon performing an orchiectomy, the kind where you're pretty sure your startup will never see the light of offspring again. Meanwhile, my soul slowly left my body and started browsing LinkedIn for consulting jobs, wondering if I was too old to become a surf instructor somewhere nice in Brazil.

Mercifully, we'd only scheduled 15 minutes. When that blessed 1/4 mark hit, I practically screamed "OH WOULD YOU LOOK AT THAT!!, LOOK AT THE TIME, GOTTA RUN!" like a teenager ditching prom with their dignity still somewhat intact. I had a very important meeting with my mirror scheduled, where I planned to stare deeply into my own eyes and whisper "what the actual fuck?"

The question haunting me was deliciously simple: What mystical unicorn were we seeing that others couldn't spot with a telescope? What neon-bright red flags were others seeing that we were blind to? Who was the crazy one here? In hindsight: My guess? Oh darling, we all were wearing different flavors of crazy pants in this circus. 

I still remember vividly the day I called my co-founder Adam Carrigan. Picture this: it's a Sunday afternoon, and I'm interrupting what sounds like one of those trendy online workout sessions. You know the type; where fit people on screens make you feel bad about your life choices while you wheeze like a broken accordion.

"Hey mate, wanna quit your cushy consulting gig?" I asked, with all the casual nonchalance of someone suggesting we grab coffee, not torpedo our careers. He responded: "Yes yesterday, but no, it depends, what's up?"

Through his labored breathing (apparently planking and important life decisions go great together), I could practically hear his brain short-circuiting. We both had great paying jobs, finally, the kind your parents brag about at family gatherings, after a year of: oh well, they work on their computer playing businessmen. Our previous startup adventure had left us with enough emotional scar tissue to make a therapist retire early. But here we were, like recovering adrenaline junkies eyeing a bungee cord, itching for another fix.

"I think databases will become enterprise AI, they will have a Mind" I declared, probably sounding like a fortune cookie having an existential crisis.

Now, Adam has the built-in risk aversion of a British man facing unscheduled tea time, we're talking about he is the friend who considers jaywalking a gateway crime to full-blown anarchy. But to my utter bewilderment, this gentleman didn't even hesitate. If my partner in business, with whom I'd previously achieved the impressive feat of turning perfectly good savings money into absolutely nothing during our last startup adventure, was ready to leap back into the circus of entrepreneurial chaos, then surely this idea must've been pure, unadulterated, 24-karat gold, right?

...Right?

After back and forths on perfectioning our thoughts. We had this wild conviction that Machine Learning would eventually evolve into applied AI, you know, AI that actually does useful stuff instead of just crushing board games, PhD papers, or writing suspiciously existential poetry about cats. We believed, with the fervor of shroomed prophets, that every company on Earth would eventually crave this magical decision-making juice like it was the last bottle of water in the desert. And the market size? For enterprise AGI? Oh darling, just every company in existence. No pressure. No biggie. Just casual world domination.

But here's the punch: this was pre-2023, when AI wasn't the corporate equivalent of avocado toast, something every self-respecting CEO had to have on their agenda. Our solution, while technically impressive (if I do say so myself), required more attention than a needy ex with abandonment issues. In retrospect, in many ways, it felt like architects trying to sell flying houses before humans had mastered the wheel. Technically brilliant, practically useless. So, at this time, after having put tears and sweat into it. For most people, still, our idea was most of the time shit or at least not gold.

A few weeks after my disastrous pitch to George, I found myself in NYC for a wedding. With the audacity of someone who's been rejected so many times they've achieved enlightened zero-fucks-given status (a state I hope you too will reach), I messaged George: "Remember that startup you verbally eviscerated? Want to grab coffee?" To my eternal surprise, he agreed to 15 minutes of coffee. At the time I did not know him, so I thought it was probably out of morbid curiosity, but; as I later learned that day: he's actually a really cool and curious guy. He just happens to have the bullshit filter of a nuclear-grade truth detector.

Until then, we'd been blessed with magical thinking investors, you know, the kind who see a couple of passionate nerds and think "Well, they'll either change the world or create a spectacular failure worth telling stories about." Later, I will write about how crucial those investors were for our business. But George? George was a product person. He saw the holes in our plan like a wine mom spots judgment at a PTA meeting.

In my notes from our first call, there was this gem: "You're crazy if you think you can compete with existing players that already have the data." We saw ourselves as market disruptors, thinking people would flock to our solution like teenagers to a TikTok trend. But convincing companies to switch data systems is like trying to convince someone to move houses just because you invented a slightly better toilet. Sure, some might bite, but most people are perfectly happy with their current crapper, thank you very much.

The way investors think is brutally simple: how likely is it that this company will become absolutely massive? The question George was probably asking himself was: "How likely is it that these adorable dorks will convince everyone to abandon their perfectly functioning data infrastructure and jump ship to their solution?" About as likely as finding a llama-corn (that rare hybrid born nine months after a llama had a magical night of passion with a unicorn) sipping a pumpkin spice latte at your local Starbucks. But then, while having my profound heart-to-heart with the mirror, it hit me like an epiphany: we didn't need to move anyone's data! We just needed to integrate with every single data source that existed. Forget being data hoarders, we'd become the master plumbers of the AI era, connecting all the pipes without stealing anyone's precious data-toilet.

As we met for coffee, I had 15 minutes and I explained the epiphany with the enthusiasm of a hippie that has discovered that they too can brew their own kombucha, all of a sudden I too saw the parts of the puzzle make sense, like when you go from staring at a Magic Eye poster for three hours seeing nothing but static to suddenly spotting that 3D dolphin that was hiding there all along, mocking your perception. The pieces started falling into place faster than Silicon Valley startups pivot to AI nowadays. It was like watching Bob Ross paint one of his "happy little accidents", starting with chaos and ending with something that actually made sense.

Before we knew it, we were walking the streets of New York way past those promised 15 minutes. And then George dropped the kind of validation bomb that hits different, the kind that makes you feel like maybe, just maybe, you're not completely delusional after all. "This idea," he said, with the gravity of someone who's seen enough startups to develop PTSD, "if you pull it off, this is how a giant business is built." Then came the classic investor one-two punch: "Now you just need to execute and validate." Which in startup speak translates to "You've got a tiger by the tail. Don't screw this up". All of a sudden a "shitty idea" started to look more like a good idea!

\setcounter{section}{2}
\section*{Rule 2: You need to grow the skin of a fucking lizard, you can't give up!}
\addcontentsline{toc}{section}{Rule 2: You need to grow the skin of a fucking lizard!}

Oh sweethearts, forget those Hollywood montages where success arrives faster than a Silicon Valley bro can say "A.I". Sure, there are some lucky bastards out there who catch lightning in a bottle and ride it straight to TechCrunch headlines, we probably hate them too. But, for the rest of us mere mortals? The pre-seed stage isn't some heroic sprint to glory; it's more like trying to survive a zombie apocalypse while wearing havaianas, don't take me wrong they feel like walking on clouds, but you can't run a block on them.

Later in Chapter 2, we will cover the basics on what you need in your backpack to survive this, soon, I hope y'all learn that this is all about how you handle risk. For now, however, you need to  understand that risk is a two way avenue; before anyone leaps and says ‘yes, let’s fucking do this’, 'yeah I will buy your product', 'where do I sign?', you must be the first-step-taker; you take the risk to ask first, and as such, If you haven't yet mastered the art of getting rejected more times than a Nigerian prince's email asking for social security numbers, don't sweat it. It's not technically required. This line of work will give you a Post Doc in rejection faster than you can say "so,.. what do you think?." Your skin will grow thicker than a rhinoceros wearing kevlar, it has to. It's like asking if a boxer gets used to being punched in the face. Who knows? Maybe. Either way, You'll build up immunity to rejection like you will build up immunity to ramen; it's not that you enjoy it, you just lose all form of taste. 


\subsection*{Rule 2.1: YOU NEED TO BE IN LOVE WITH THE PROBLEM!}

This sub-rule which is the opposite of not giving up rule, is so painfully obvious that as I write it, it makes me want to staple my own eyelids shut, but here we are, having this conversation like two people trying to assemble furniture while high. It's one of the most important rules to follow if you want to survive this madness. \textbf{If you can't see yourself doing this for the next 10 years, don't do it.}  If you can't picture yourself still doing this particular startup/thing when your future self is explaining to their therapist why you named your firstborn "Series A," then for the love of all that's holy and cute, stop right now. Go back to your gig where the biggest risk is accidentally CC'ing your boss on an email about how much you hate corporate Taco-Tuesdays. And I don't mean that in the sense that you need to always be the kind of person who can look at a spreadsheet full of red numbers and think "challenge accepted" rather than "where's the nearest bridge?". I mean that if the problem, the industry that you picked to solve is not something you would want to solve for the next many years, don't do it. \textbf{Life is in fact short.}

\subsection*{Let me tell you about my face-plant that relates to this whole circus:}

Picture this: Adam and I, we were just about to crash and burn with our previous hardware startup that we failed at (Real Life Analytics). We were broker than a piggy bank at a hammer convention, questioning every life decision that led us to believe hardware was a good idea. (Pro tip: Hardware is like dating a supermodel, looks amazing, but the costs of that relationship can kill you). If you've succeeded in hardware, you are my hero, you can and you should look down on software people like they're playing with Fisher-Price toys while you're wrestling bears. It's not even apples and oranges; it's more like comparing one of those walkers that toddlers use to learn their first steps to a rocket.

So there we were, desperate enough to email random people on LinkedIn. By some miracle, this prime-cut hardware-focused fund invited us to San Francisco. I mean,... they didn't pay for our flights, hotels or anything, they just said, if you are in the Bay we can meet in our office, so clearly East coast to SF was a stone throw away, so yes, we were in the area, and we pitched our hearts out to a partner whose face had all the emotional range of Poker world champion. If we'd pitched to actual concrete, we might've gotten more emotions from it. They did ask good questions, and by then we had answers smoother than a VC's LinkedIn humble-brag about their latest unicorn, but our souls had already left the building.

Meeting over, Adam and I stood at that, some random pier where their office was, staring at the bay like two rejected contestants from The Bachelor. "So... this is the end, right?" Adam shared with a smirk, but also with all the enthusiasm of someone announcing their own funeral. "What shall we do?"

Our hearts were more broken than a developer's code after a coffee spill on their only backup. We knew that their "this is interesting you guys, we'll reconvene and message you tomorrow" really meant "Thanks for flying across the country you bosons, but we could've crushed your dreams over the phone."

And then we did what any self-respecting failed entrepreneurs would do: we decided to get absolutely hammered good-bye. Like college freshmen with their first fake IDs, we bought the cheapest booze we could find and went to town.

Next morning, we woke up on our friend's Icelandic ice breaker ship (because believe it or not that's a thing people have in San Francisco), splitting headaches and phones buzzing like angry wasps. I answer, and it's no other than the investors from yesterday: "Hey guys" -this time all emotions- "we'll be in the office if you want to discuss investment terms."

I jumped up faster than a cat in a field of mirrors, found Adam, and we sprinted to their office. Only just before we arrived, looking like we'd just lost a fight with a seafood buffet, smelling like if we had showered in lobster juice, did I realize we probably shouldn't have given up so early. There was still one bullet in the chamber, but we'd already written our own obituary yesterday with ink made out of tequila, we knew we could not see ourselves doing hardware for advertisemnt in the following years. The solution we loved, it was what two nerds like us would enjoy building for fun, but we were looking for a problem to our solution, we couldn't care less if people did or not see the right billboard ad on their way home.

I called back and declined the meeting, which probably made me wonder if we were the first founders in Silicon Valley history to turn down money because their hangover felt like their brain was being used as a drum set by a toddler that just discovered sugar. But here's the real lesson: we had already emotionally checked out. The day before, we'd basically done the startup equivalent of not just burning bridges, but nuking them from orbit, tap dancing on the ashes, and sending a postcard about it. We'd mentally resigned harder than someone who quits their job by storming into their boss's office with an opening line such as "I am glad your mother did not swallow you, but... ", proceeds skywriting with their middle finger "I QUIT"  and then parachuting into their exit interview wearing a banana costume. The desire to solve that particular problem was born more dead than my will to ever drink tequila or whatever that elixir was that we killed our last intentions in.

For years, I wondered what could've been if we had taken that last meeting. But now I know it was for the best. We needed that failure like a teenager needs embarrassing yearbook photos. It taught us the most valuable lesson: Before you start taking checks from people, It's ok to stop, if you don't love the problem your startup is trying to solve, on the other hand, if you see yourself working on this for a very long time, never give up, because sometimes, just sometimes, your lowest moment is right before someone decides to bet on your crazy dreams. This lesson came very handy as you will see in Rule \#3.

\setcounter{section}{3}
\section*{Rule 3: Time Is Money— And You're Bleeding Both}
\addcontentsline{toc}{section}{Rule 3: Time Is Money— And You're Bleeding Both}

As you progress through the various stages of pre-seed funding, which we will cover in Rule \#4, the dance with time becomes a knife fight. Investors at this level value speed as much as you do; they aren't looking to become pen pals. Time is the ultimate currency, and at this stage, it's a cruel and impatient god. For pre-seed, the funding time metrics are stark: one call should be enough for tens of thousands. One or two calls or meetings should be sufficient to pull in hundreds of thousands. To hit the millions, three or four meetings max. If you find yourself deep in a labyrinth of endless investor lunches, long email chains, and vague promises, stop. You're stroking the wrong cat, and both sides are wasting precious hours. The best investors at this stage know the value of quick decisions—either they're in, or they're out. Dragging it out signals either a lack of conviction or a mismatch, and in the pre-seed world, hesitation is as fatal as outright rejection.

Our fable to this rule, is a near death experience. Six or so months into our new startup (MindsDB). Chon Tang from Skydeck (you can research about him, but essentially like Professor X to the X-Men Academy for startups), our first investor and now friend. This guy had front-row seats to our fundraising circus: watching us pitch to everyone from venture capitalists to that one guy who looked suspiciously like a janitor but we still believed they could be our angel investor, maybe. We were like desperate singles at last call, hitting on every wallet in sight.

But holy quantum algorithms, were we working harder than a hamster that found a bag of speed next to a bag of nuts! There's something absolutely intoxicating about building a product from nothing but the fumes of your hallucinations while simultaneously trying to figure out what the hell a "go-to-market strategy" is. Chon met with us for countless hours, probably questioning his choices that one moment he said, 'welcome to Skydeck', but sticking around anyway. He was what every entrepreneur needs at that stage: a "fuck it, let's do it" bouncing wall who'd listen to our fever dreams with the patience of a Buddhist monk at a heavy metal concert.

Because here's the thing about the early startup game: the more lost you are, the more time you can spend debating the stupidest of things, like number of pixels for the left margin of paragraph four on your website, than actually building something and trying it with the market. But with Chon, we were all about trying things like teenagers discovering cooking for the first time: throwing oil-drenched spaghetti at the wall with the enthusiasm of a blockchain newb at a Web3 conference, completely oblivious to the fact that all that oil meant nothing would ever stick. To the outside eye, we were basically conducting a masterclass in how NOT to make pasta while thinking we were the next Gordon Ramsay of startups, but from the inside we were working harder than ever to find that one spaghetti we did not put oil in..

The market wasn't ready for us, or maybe we were too busy being the proud parents of our "super cool solution" to notice that -businesses actually- need, you know, business fundamentals. Details, etc.

So Chon invites me to lunch, and I'm thinking "free food!" but no, it's intervention time. He's giving me the "you fought a really good fight" speech to someone that can't open their eyes from how swollen the wounds are, one more punch in the face it's not worth it, people already saw you being beaten to death, they just want you to be able to walk a few months later. He knows we tried harder than a vegan trying to convince Tyson Foods board about broccoli's exciting flavor profile.

There I am the next day, packing up our office into a sad little box (which mostly contained empty Red Bull cans), and just like that is when fate decides to play one of its practical jokes. Some other team, the darling of our cohort, cancels their investor meeting, because they can't say yes to everyone, and suddenly I'm the backup of the backup dancer called to center stage. 

I show up to meet this investor who's crankier than a cat in a bathtub about driving all the way to Berkeley from god knows Santa-somethin' for a cancelled meeting. He suggests a phone call instead so he can beat traffic across the bridge, and I'm like "Sure, one more bullet in the chamber!" I call and pitch with the confidence of someone who wasn't packing their failure into a cardboard box as they spoke, channeling everything opposite to that time Adam and I fumbled the bag in San Francisco way back then. Because now, I know it's not over until it's over.

The phone call ends with that classic investor line: "Aight, send me your terms and deck, I'll look at it while driving." Translation: "Cool story bro, I'll file this under 'Never in a Million Years.'" But remember Rule \#2? Never give up? Well, we sent that pitch deck faster than he could put his blinkers on, complete with a signed YC Compatible SAFE agreement (see the glossary for more details about SAFE agreements) and our bank wiring detailed instructions next to the "we will make you rich and looking forward to catching up again soon".

Fast forward a week or two, because time flies when you're busy writing startup obituaries, and my phone rings. This investor, sounding like someone who just discovered an untouched pizza in their fridge after a rave, asks if we've received the transfer they made days ago. Just like that, like a zombie startup that learned to tap dance, we moonwalked right out of the startup graveyard. And here we are, telling the tale like contestants who not only survived the Hunger Games but also managed to steal President Snow's Netflix password in the process.

So what did we learn? Well, we learned something so fundamental it should be taught in preschool right after "How to Sign Your Name on Crayon Masterpieces 101." Picture this conversation:

\begin{quote}
"What did you learn today, gummybear?"

"I drew this monkey, Daddy! And I learned that you shouldn't waste a single precious millisecond extra with early-stage investors that are not into you!"

"That's my little girl! Your life expectancy just increased by 40\% if you ever decide to be one of them interpernurs. And, I must say, that monkey is absolutely adorable."
\end{quote}

You see, the brass knuckles of this tale weren't just about our stubborn refusal to give up. No, it was about how we became the startup equivalent of that person who thinks is your ex, just because you exchanged a few flirty lines, and since then, just won't stop texting. We haunted investors until they came back with messages like: "Look, I was planning to ghost you, but before I file a restraining order, let me be crystal clear: It's not a maybe, it's a hard no. Not now, not ever, not in this universe or any parallel dimension where unicorns tap dance on Mars to wallabies playing in a band. I've got a million other deals I haven't met yet, and your constant messaging is making my phone vibrate off my damn desk."

And those times they asked for due diligence? were they interested? well we were on idea stage, so more early stage than a caterpillar's first identity crisis? That was just them using us as their personal market research interns. They knew it was all speculation at this point; like trying to predict which way a drunk penguin will waddle. But hey, at least we learned something new every time we jumped through their hoops like circus seals chasing premium fish.

So what's the protocol when investors start asking for your grandmother's DNA and a detailed analysis of your childhood trauma, all things outside of your deck and investor room folder? Here's the deal: ghosting is like a dance-off at a startup mixer; it takes two to tango, baby! But trust me, your response should NOT depend entirely on your dance card. Are you the belle of the ball with VCs throwing term sheets at you like confetti, or are you the awkward wallflower who just got their first "maybe" after 47 rejections?

Either way, channel your inner time management guru and remember: your time and their time is worth its weight in Bitcoin (circa whenever it's high as Friday night Snoopdog). Sure, if you want their money, you can play along and say "Absolutely, I'll send you whatever documentation your heart desires!" But here's the twist: make it conditional on a SAFE agreement. The, show me yours and I will show you mine paradox. That means they commit to investing by giving you a SAFE, if they like what they see, then they wire the money, and on the other side you commit to taking their money if their due diligence doesn't reveal that you're actually three raccoons in a trench coat. It's way better than becoming their unpaid market research intern while they dangle funding like a cat toy in front of an increasingly desperate kitten.

Bottom line? I wish we'd known from day one that the formula was simpler than a one-piece puzzle: Take the call, read the room faster than a poker pro, learn, improve, and move on. Send that deck, SAFE agreement, and bank details with the confidence of someone ordering pizza at 3 AM. I'm convinced our savior investor only wrote that check because they thought, "Well damn, anyone ballsy enough to send their bank details after one phone call must be either crazy or brilliant, and in Silicon Valley, that's basically the same thing." 


\setcounter{section}{4}
\section*{Rule 4: Avoid the Wrong Investors}
\addcontentsline{toc}{section}{Rule 4: Avoid the Wrong Investors}

Hear me out and repeat: \textbf{``Only take money from people who can laugh while watching it evaporate.''} Not because you're planning to fail---you'll work harder than a rat on a cocaine-coated wheel---but because this is pre-seed, baby. The odds against you are about as friendly as an Australian Magpie in mating season. Pre-seed is not a garden; it’s a battlefield. It’s where you sharpen your teeth on the grindstone of doubt, where your idea sheds its soft, useless parts and becomes something sharp enough to cut in real interest. And before you learn what this means, you want to understand who you want to---but most importantly: who you \emph{don’t} want to draw interest from.

\subsubsection*{Warning: Friends, fools, and family round}
Some call pre-seed that, and if you ever hear this advice, heed me: \textbf{run}. Run like hell. You are not a con artist peddling snake oil under the guise of a brilliant idea!!! \textit{Nein!} As just explained, and worth diving deeper: The pre-seed stage is the riskiest chapter in a startup’s life cycle, a place where even the most determined founder is more likely to fail than to thrive. This is where the unvarnished truth comes out: (once again, repeat after me, 5 times) \textbf{``DO NOT take the money that grandma saved for your mother’s rainy day!''} At this stage, when you can’t tell your ass from your hand, that money is better off in the S\&P 500 or parked in some safe, diversified corner of the financial universe.

The sad reality is, believe it or not, and god bless them, there’s no shortage of good souls, eager to believe and eager to part with cash they can’t afford to lose. It might seem tempting when you're desperate to make your idea fly, but taking someone’s life savings, money they bled years to accumulate, is wrong, and unless you are an asshole, it is the fast track to sleepless nights, moral bankruptcy, and the guilt that will haunt you long after the startup burns out. The path is lined with those willing to bet everything, from the trust fund artist looking for meaning to the retiree with savings that shouldn’t be gambled on a pipe dream. These are the ones you must resist. In the pre-seed stage, don’t take money that isn’t designed to risk disappearing. It’s not just about doing the right thing---it’s about ensuring that the support around you is rooted in smart, informed decisions. At this stage you cannot be here to be someone’s miracle or savior; you’re here to build, to prove, and to play the long game with the right backers who understand exactly what they’re in for.

\subsubsection*{Warning: You'll meet the "advisors"} 
Those beings who offer their "expertise" in exchange for equity. They'll promise introductions that never materialize, like crypto gains in a bear market. When they come at you with their advisory agreement PDFs, hit them with the ultimate uno reverse card: "Love your enthusiasm! How about you put some skin in the game? A small check would really align our interests..." Watch them disappear faster than free pizza at a developer meetup or you will find what you want which is that first round of checks.

\subsubsection*{From whom then?}
Acknowledge that for the true investor, the pre-seed stage is a numbers game, and it demands a certain kind of participant---two, in fact: the informed and the strategic. They both spread their bets thinly across many hundreds of startups, knowing that most will fail but one or two will soar high enough to cover the losses many times over.


\setcounter{section}{5}
\section*{Rule 5: IT IS NOT a Single Round; It's an Ongoing Round}
\addcontentsline{toc}{section}{Rule 5: IT IS NOT a Single Round; It's an Ongoing Round}

You've started on your tighty whities, standing at the edge of a financial cliff, wearing nothing but your dreams and a pitch deck. The wind of market uncertainty is whipping through your metaphorical hair, and somewhere in the distance, an Angel is probably saying "interesting" while mentally planning their dinner. 

Here's where the illusion shatters: raising at pre-seed isn't a dramatic all-or-nothing moment with champagne and applause. It's an ongoing battle, a financial siege. Investments at this stage are a mosaic of SAFEs and convertible notes, pieced together as you scramble for validation and resources. You probably won't get one big, gleaming check; you'll get trickles, possibly starting with breadcrumbs that keep you alive while you scrape for the next. It's not linear; it's an unpredictable patchwork. And here's the paradox: as your idea becomes less terrible, your terms get better. What starts with a few tens of thousands can swell into millions if you prove that you're more than a flash in the pan.

The beautiful absurdity of it all: in the beginning, everyone's waiting for someone else to make the first move. It's like trying to start a conga line at a particularly stiff corporate party: nobody wants to be the first to wiggle their hips. But once a few people join in, suddenly everyone's an enthusiastic dancer.

The psychology works like this: Walking into a room announcing "I'm raising a few million" with empty pockets is like trying to sell timeshares on Mars. But watch how the story transforms: "We're raising 100k and already have 50k", now you're selling tickets to dudes into a party that's already buzzing with chicks on a bachelorette party from Sweden. "Raising 300k with 100k in the bank", the intrigue builds. "Looking for 1M with 300k-600k locked in", if this happens quick enough, suddenly you're the hottest ticket in Silicon Valley.

Now, as we dive in details, here comes the delicate art of terms (keep them YC-safe compatible, unless you enjoy watching sophisticated investors give you sudden and intense migraines or delusions). The "cap" (look it up): your theoretical valuation, is like a price tag in a parallel universe where crypto only goes up. It's a number that exists in the realm of "trust me, bro" until enough people nod sagely and say "seems legit."

Start with a valuation that makes early investors feel like they're getting in on the ground floor of the next big thing. Then, as more checks materialize, gradually adjust that number upward like a master chef adding salt; carefully, tastefully, but with conviction. Check Crunchbase or make a friend that will let you peek at their Pitchbook to see what other visionaries in your space are commanding. It's like a sophisticated game of pricing psychology where everyone pretends the numbers are based on complex mathematical models, when really it's more like interpretive dance. Welcome to the pre-seed evolution!, where your funding journey transforms in stages like a Pokemon with anxiety issues. 

\subsection*{Stage 1: The "Please, Anyone?" Phase (\$10k-100k):}

This is where you're basically selling hope wrapped in a PowerPoint. Angels look at you the way a parent looks at a child trying to explain why drawing on the walls is actually not good for property values. But here's the thing: you only need one believer, one beautiful optimist who sees past the duct tape and dreams to your actual potential.

\subsection*{Stage 2: The "Hey, We're Kind of Real" Phase (\$100k-200k):}

Suddenly, you have that magical first check. It's like getting your first kiss; awkward, probably could have gone better, but hey, someone thought you were worth it! Now you can walk into rooms saying, "We've already raised..." which is startup speak for "See? Someone else thinks we're not completely crazy!"

\subsection*{Stage 3: The "Now We're Cooking" Phase (\$200k-500k):}

This is where the momentum builds faster than technical debt in a hackathon. Each "yes" breeds more "yeses" like rabbits discovering Viagra. Investors start responding to your emails faster than they ghosted you in Stage 1. You're no longer pitching; you're "updating interested parties on your round's progress." Fancy!

\subsection*{Stage 4: The "Is This Real Life?" Phase (\$500k+):}

Welcome to the line to the tickets for the entry leagues. You are raising \$1m and you already have \$500k, And now you know how this game works, you can go up to a few million this way. You can choose to dance this conga to your own playlist, but here once again, a lot of this dance is accelerated if you join an accelerator, and if you do, of course, please do it right, because there is a wrong way for sure.


\subsection*{So, what is the difference between a pre-seed stage and a seed stage?}

We will cover Seed stage and Seed rounds in chapter 3 and 4, as well as in the Glossary. But understand that there black and whites are not a thing in early stage. For you, you are in a journey of building a company, this language, pre-seed, seed, series A, etc, its fundraising lingo. For now, let's just say that what you can take away from this chapter is that this language helps you set the expextations for when you are asking for money.

\subsubsection{I am in PRE-SEED stage?} 
When you tell investors that you are in the pre-seed stage, the good investors (meaning those who are used to investing in early-stage startups) will know that they are making investments based on how promising the opportunity looks just by examining the problem you are trying to solve. They will decide whether to invest based on the two risks we cover in Chapter 2: Market risk and Founder risk. The checks you receive will not be given at a set valuation, but rather with the prospect of a future valuation. This means that, at this stage, no one can really say how good the opportunity is, because you are still selling a vision rather than a proven business. They are literally making a bet—please read the glossary for details about convertible notes and SAFE agreements. Essentially, your value is unknown; you don’t even have a product yet, but you are convincing people that the problem you are setting out to solve is worth investing in. If you succeed, they hope to turn a relatively small check (tens of thousands to a few million dollars) into a significant return. Investors need to be able to imagine you as someone who could build a wildly successful business (that means picturing you with one to ten horns on your forehead and shitting rainbows). 

So, if you are talking to an investor whose LinkedIn profile says “later stage, (A,B,\,$\Omega$)” you are barking up the wrong chickens. The best thing you can do is ask them for an introduction to someone who invests at the early stage. That introduction will help you get connected in the right direction.

\textbf{What is pre-seed money good for, you may ask?} Well, it's to pay the team, to cover operational costs, and ideally to get you to a point where you can focus solely on customers, building, and validating your solution. So eventually, having 18–36 months or so of runway ahead of you without having to spend all your time fundraising is, for the most part, what a SEED round will likely provide.

\subsubsection{So, am I in SEED stage?}

Well, this is something that happens a bit more organically than you might think.. you stumble into seed stage. The moment someone wires you money, you’re “investor-backed”—which sounds glamorous until you realize it’s just you, a spreadsheet, and a redbull can at 2 a.m. The important thing to know is that as you collect them signatures on SAFEs, you must, in tandem, like a juggler of shit-on-fire, you shall perform the ancient art of “progress.” This means transforming your idea from “wouldn’t it be cool if…” to a “look, it actually solves something!”. This metamorphosis is not overnight. Your business will slowly ooze from “pure fantasy” to “mildly convincing hallucination”.  Suddenly, you’re not just selling air in a can—you’re selling oxigen for oxigen deprived, with a label and a barcode. That’s when early-stage investors start muttering, “Maybe it’s time for a priced round,” like wizards debating whether it is time to summon a demon or to give you sparklers until you grow a pair. We’ll unravel that particular demon spell in Chapters 3 and 4, and in the glossary, because you’ll need a talisman.

So, If you’re not hearing these magic words: "Fuck it, how about we do a priced round?", yet. Focus on building your business, and—just as importantly—on protecting your own and your company’s collective butts from the two cosmic banana peels we discuss in Chapter 2. Until then, set expectations low and keep telling people you’re in the pre-seed stage. 

