\chapter*{Glossary 1 - Pre-seed Startup Lingo}
\markboth{GLOSSARY 1 - PRE-SEED STARTUP LINGO}{GLOSSARY 1 - PRE-SEED STARTUP LINGO}
\addcontentsline{toc}{chapter}{Glossary 1 - Pre-seed Startup Lingo}


\section*{Value {\normalfont\textit{| ˈval-yü |}}}
\addcontentsline{toc}{section}{Value}
{\setlength{\parskip}{0pt}%
\textit{noun-verb  [founder units]}%
}

\vspace{1.5em}
 

\textbf{Value is a mental proccess}
- Repeat this ten times-: Value is the mental proccess that someone goes through when they look at something and compare it to their needs and wants. In this book, those needs and wants are to solve a problem they have. 

\section*{Idea {\normalfont\textit{| ī-ˈdē-a|}}}
\addcontentsline{toc}{section}{Idea}
{\setlength{\parskip}{0pt}%
\textit{concept  [sanity chewer]}%
}

\vspace{1.5em}
\textbf{Your idea is to solve a problem} - Here’s the only tattoo-worthy advice in this book: Every time your brain whispers, “I have a business idea,” slap it and yell, “No! I have a problem to destroy!”. \textbf{If} your “idea” isn’t a desperate attempt to fix something that’s been chewing on your sanity like a raccoon in your attic, you’re in urgent need to get back to the drawing board.}

\section*{Problem {\normalfont\textit{| ˈprä-bləm |}}}
\addcontentsline{toc}{section}{Problem}
{\setlength{\parskip}{0pt}%
\textit{noun  [cheek agony]}%
}

\vspace{1.5em}

\textbf{Someone's royal pain in the ass} - That someone can be you, if not, if you don’t wake up cursing the problem you want to solve, If you don’t have the problem; go find someone who does. Befriend them, interrogate them, buy them tacos, whatever it takes. If they run away, good—chase another. Repeat until you find a pain so real, like a Valdemort, where people would Venmo you just so you don't say the problem's name. That’s the only “idea” that matters. Everything else is just startup cosplay.

\section*{Founder {\normalfont\textit{| ˈfau̇n-dər |}}}
\addcontentsline{toc}{section}{Founder/Entrepeneur}
{\setlength{\parskip}{0pt}%
\textit{noun  [problem smasher]}%
}

\vspace{1.5em}
\textbf{You, I suppose} - The person that sees a problem and is crazy enough to think that they can solve it.
You must, if you are reading this book see solving the poblem as an eventual but inevitable exchange of value, this means that eventually in the journey of solving the problem, you will be exchanging solving the problem for money. You must believe this is as true as gravity. Otherwise burn this book.


\section*{Runway {\normalfont\textit{| ˈrən-wā |}}}
\addcontentsline{toc}{section}{Runway}
{\setlength{\parskip}{0pt}%
\textit{noun  [survival countdown]}%
}

\vspace{1.5em}

\textbf{How long you can survive without raising more money?} – Runway is the number of days/months/years between now and “the moment you start eyeing your houseplants as potential salad” It’s simple math: take the cash in your bank, divide by how fast you’re burning it, and voilà—that’s how long you have before you’re pitching VCs from a Starbucks bathroom. The initial goal is to prolong this timeline until you find product-market-fit, which in the Bay Area you can do thanks to “early stage funding.” 

\section*{Silicon Valley {\normalfont\textit{| ˈsi-lə-kən ˈva-lē |}}}
\addcontentsline{toc}{section}{Silicon Valley}
{\setlength{\parskip}{0pt}%
\textit{proper noun  [about south]}%
}

\vspace{1.5em}

\textbf{Back in the Day, It Was It} – Once upon a time, if you wanted to build a startup, you had to pilgrimage to Santa Clara, California, where the air was thick with WiFi and the scent of burning venture capital. But then, like a fungus, VCs and founders oozed out across the Bay Area, gobbling up real estate and artisanal toast until the whole San Francisco Bay Area became one giant, overpriced tech terrarium. Now, if you so much as sneeze out a tech-business-idea anywhere from Palo Alto to Berkeley, someone in a Patagonia vest may try to invest in your allergies.

\section*{Bay Area {\normalfont\textit{| ˈbā ˈārē |}}}
\addcontentsline{toc}{section}{Bay Area}
{\setlength{\parskip}{0pt}%
\textit{proper noun  [Vegas for nerds]}%
}

\vspace{1.5em}

\textbf{Where hoodies are formalwear} - To those who haven't penetrated its mysteries, it is easy to picture a Las Vegas casino at 3 AM, that liminal hour when reality starts to fray at the edges. The air thrums with an electric drone, time pools like mercury under neon signs, and cigarette smoke performs lazy arabesques through shafts of artificial light. At the roulette table, a man radiating borrowed confidence scatters his chips across the felt like casting bones for divination—black, red, 20, 47—trying desperately to impose pattern upon chaos. The wheel spins with cosmic indifference while the gathered crowd holds its collective breath, united in that peculiar communion of chance. Most of these pilgrims to the temple of probability stumble away with emptied pockets and hollowed eyes, dialing familiar numbers that bridge the gap between delusion and reality: "Mom? Yeah... I need a ride." But there's always that singular anomaly, that statistical outlier. She glides through the casino doors on stiletto heels that tap out victory in Morse code against marble floors, her smile gleaming like a blade in the desert dawn. She beat the house! The chosen one ascends!

The Bay Area may appear as a more sophisticated version of this tableau—a high-stakes playground where dreams are currency and reality bends like light through fiber optic cables. Some dismiss it as Vegas for the cognitively enhanced, players buzzing on a cocktail of ambition, Adderall and Ketamine. But beneath this superficial chaos, there are people that are playing an entirely more complex game, it's a place where people that want to solve problems with technology meet with risk gurus that manage the highest concentration of "fuck-yeah-capital" in the world, if you are reading these pages, the Bay Area is a place you want to be in.



\section*{Early Stage {\normalfont\textit{| ˈər-lē ˈstāj |}}}
\addcontentsline{toc}{section}{Early Stage}
{\setlength{\parskip}{0pt}%
\textit{phase  [reality dodging]}%
}

\vspace{1.5em}

\textbf{The startup hunger games} - Let me lay it out straight: until you're pulling in a sweet, sweet mill to ten-mill in annual recurring rev. The kind that makes investors purr like well-fed kittens. 
You're still crawling through the trenches of pre-seed and seed funding, where you're not selling facts (you don't have any); you're selling futures. Your data isn't really data: it's poetry written in the language of possibility—a story about tomorrow that you're asking others to believe in today. Don't waste your time crafting beautiful Excel prophecies; they're about as useful as broccoli for vampires. Your job isn't to prove you're right about exact numbers; your job is to convince yourself first, then your team, and then the rest of the world that this is an idea worth investing time, resources, and money into. At first, it will be 100\% storytelling, but as you progress, it will become a mix of that and what you are learning from your ever-evolving Minimum Viable Product.

\section*{Pre-seed Stage {\normalfont\textit{| ˈprē-ˈsēd ˈstāj |}}}
\addcontentsline{toc}{subsection}{Pre-seed Stage}
{\setlength{\parskip}{0pt}%
\textit{phase  [mom's nightmare]}%
}

\vspace{1.5em}

\textbf{The “I have an idea, please clap” phase} - Some call it idea stage, some call it pre-seed round stage, your mom calls it "please don't quit your job mijo". When it comes to fundraising, Pre-seed stage is the when you are at the first round of checks for your fever dream of a company. When you have an idea (a problem) and you are trying to validate that it is a problem worth solving. (For more details, please read Chapter 1, rule \#5).

\section*{Seed Stage {\normalfont\textit{| ˈsēd ˈstāj |}}}
\addcontentsline{toc}{subsection}{Seed Stage}
{\setlength{\parskip}{0pt}%
\textit{phase  [squirrel dreams]}%
}

\vspace{1.5em}

\textbf{The “look Ma, I am not a homeless person anymore!” stage} – When your startup finally gets enough cash to stop stealing WiFi from Starbucks, but you’re still living on hope and caffeine.
Seed stage is when you have started working on solving the problem and you are trying to validate that the solution you have been cooking up, has a growing number of people say: damn, I have this pain and I am trying this solution of yours and it is working. This doesn't necessarily mean that you have a product-market fit, and that people are paying for it, but it means that you are on the right track. Now hear me out, if you can get people to pay for your solution, the earlier the better, but to raise a seed round, it is not required that your bank account is fatter than a squirrel in a peanut butter factory, It would ALWAYS be nice though. (For more details, please read Chapter 1, rule \#5).

\section*{Series-A Stage {\normalfont\textit{| ˈsir-ēz-ˈā ˈstāj |}}}
\addcontentsline{toc}{subsection}{Series-A Stage}
{\setlength{\parskip}{0pt}%
\textit{phase  [sly dog territory]}%
}

\vspace{1em}

\textbf{The cash flow celebration} - Not covered in this book. But you are making cash and cash is growing as we read this one line  sentence: "Good for you ;)".


\section*{Priced Round {\normalfont\textit{| ˈprīst ˈrau̇nd |}}}
\addcontentsline{toc}{section}{Priced Round}
{\setlength{\parskip}{0pt}%
\textit{round  [valuation time]}%
}

\vspace{1.5em}

\textbf{The actual buy shares moment} - A priced round is when a VC firm says they are interested in investing in your company, and they will give you money in exchange for shares (and they mean, right now, not in the god knows when future like in a SAFE agreement). More about this in Chapter 3 and 4.


\section*{Term Sheet {\normalfont\textit{| ˈter-m ˈshēt |}}}
\addcontentsline{toc}{subsection}{Term Sheet}
{\setlength{\parskip}{0pt}%
\textit{document  [investment terms]}%
}

\vspace{1.5em}

\textbf{The “so, are we exclusive now?” document} – When someone drops the phrase “term sheet,” it means you’ve leveled up from “let’s grab coffee” to “let’s define this relationship”—with a priced round. A term sheet is the official list of rules for your investment romance: who gets what, who calls the shots, etc. However, at pre-seed, you’ll mostly hear about SAFE agreements (the startup equivalent of “let’s just see where this goes”), but it’s good to know term sheets exist. We’ll spill all the juicy details in Chapter 4.

\section*{Convertible Notes {\normalfont\textit{| kən-ˈvər-tə-bəl ˈnōts |}}}
\addcontentsline{toc}{section}{Convertible Notes}
{\setlength{\parskip}{0pt}%
\textit{agreements  [debt]}%
}

\vspace{1.5em}

\textbf{Really weird lending contract} - When you hear the words "Convertible Note," you should respond: "Yes, Ma'am, I am aware of them. We don't do Convertible Notes, but we do use SAFE agreements. By that, we specifically mean we take money using the YC SAFE agreement." Please, ladies and gentlemen, keep it standard and stick to the basics, or you are going to have a bad time later.



\section*{SAFE Agreements {\normalfont\textit{| ˈsāf ə-ˈgrē-mənts|}}}
\addcontentsline{toc}{section}{SAFE Agreements}
{\setlength{\parskip}{0pt}%
\textit{agreements  [startup prenup]}%
}

\vspace{1.5em}

\textbf{The startup world's prenup template} - A SAFE is a document that outlines the terms of your early-stage investment. Listen: every founder since 2013, who’s ever hallucinated a unicorn after their third Red Bull has used a SAFE—so unless you’re Picasso with a law degree, don’t try to get creative here. Study the damn thing, sign it, and get back to building your fever-dream empire before you start sculpting term sheets out of ramen noodles.

\section*{Valuation cap {\normalfont\textit{| ˈvā-lü-ā-shən ˈkāp |}}}
\addcontentsline{toc}{subsection}{Valuation cap}
{\setlength{\parskip}{0pt}%
\textit{number  [valuation]}%
}

\vspace{1.5em}
\textbf{The near future valuation of your company} - Before you do anything reckless, go read up on the YC SAFE agreement. But here’s the bit that actually matters: the “cap” is the imaginary number everyone pretends your company will be worth in the future. It’s not what you’re worth today (which is probably somewhere between “slide deck” and “wishful thinking”), but what you might be worth in a few months—if you’re lucky—or a few years, if you’re starring in a slow-motion train wreck. When people eventually want to do a priced-round, those early SAFE investors get to convert at the cap in their SAFE—like they found a cheat code for your company’s stock. If your company’s worth more than the cap, they look like clairvoyant geniuses and you will look like a great negotiator. If it’s worth less, they don’t get ripped off—they just pay what the new, wide-eyed investors are paying. It’s actually fair: you’ve got nothing but a pitch and a prayer, and they’re betting on your ability to turn caffeine and anxiety into a business. If your startup collapses into a flaming pile of “oops,” they lose their money, you get to walk away, and nobody’s coming to break your kneecaps.

\section*{Cap Table {\normalfont\textit{| ˈkāp ˈtābəl |}}}
\addcontentsline{toc}{section}{Cap Table}
{\setlength{\parskip}{0pt}%
\textit{document  [ownership]}%
}

\vspace{1.5em}

\textbf{A spreadsheet that lists who owns what percentage of the company} - Picture a cap table at pre-seed like a kindergarten seating chart: it’s just you and your co-founders writing down who gets the blue crayon, while investors are still outside, noses pressed to the glass, waving their SAFEs like golden tickets that don’t actually get them inside. No one else than you, your co-founders, and maybe your grandma if you ignored rule #4, owns real shares yet. But beware: the future moment when you do a priced round, your cap table mutates faster than a gremlin in a rainstorm, and suddenly every SAFE-holder is demanding a seat at the grown-ups’ table. So pretty please, keep your pre-seed cap table cleaner than a monk’s search history before nirvana, or you’ll be untangling equity knots until your next startup is a therapy app for traumatized founders.

\section*{Board of Directors {\normalfont\textit{| ˈbōrd ˈəv ˈdī-rəktərz |}}}
\addcontentsline{toc}{section}{Board of Directors}
{\setlength{\parskip}{0pt}%
\textit{group  [leadership]}%
}

\vspace{1.5em}

\textbf{A group of people who are responsible for the overall direction of a company} 
- At pre-seed, your “board” is just you and your co-founders arguing over who forgot to pay Zoom again. Please, guard those seats like they’re the last bag of Hot Cheetos in a middle school cafeteria. Any investor demanding a board seat this early is clueless. Real ones know that board seats are for priced round investors, when the drama gets good.


\section*{Investors {\normalfont\textit{| in-ˈves-tərz |}}}
\addcontentsline{toc}{section}{Investors}
{\setlength{\parskip}{0pt}%
\textit{players  [money whisperers]}%
}

\vspace{1.5em}

\textbf{The risk philosopher} -  They're pattern-recognition savants who've transformed spectacular failures and meteoric successes into a kind of muscle memory. For them, risk isn't some mystical force to be feared or worshiped; it's as fundamental as gravity, as quotidian as breathing. Their real fascination lies in watching how you, thou founderi, dance with these inevitable dangers, how you'll navigate the storm clouds they can already see gathering on your horizon.

Here's your first koan: the best investors, those rare beings who can in fact help you transmute your toilet-born prophecy into market reality, aren't mere dispensers of capital. They're risk philosophers, operating on intuition distilled from countless cycles of creation and destruction. They don't need spreadsheets or calculators; their scars tell better stories than any Excel model. They've witnessed countless founders like you chase mirages straight off cliffs, each believing they were different, special, chosen. Every dollar they commit is a vote cast into the void, and the house odds would make even a Vegas bookie blush.

\section*{Angels}
\addcontentsline{toc}{section}{Angels}

These are individuals who understand what they’re getting into when it comes to investing in startups — They are investors who’ve allocated a slice of their portfolio to bets that they know may turn to ash. To them, your venture is one high-stakes chip in a broader game. It says somewhere in the bible; YOU SHALL NEVER give an Angel investor board rights. You want these Angels to be people  that can give you: Money and connections to either more money or customers. Don't forget that the real Angel investors, by virtue of how this game being a numbers game, they don't have much time to spend in each company.  So be in the lookout for red flags.

\subsection*{Angel Money} Here's where things get interesting, like a game of hot potato where nobody wants to be the first to catch. These angels—the clever little honey badgers—they're professional followers, masters of the "you first" dance. They're not going to write you a check just because you've got a slick deck and practiced your pitch in front of a mirror until your reflection started giving you feedback. They're like those people at a party who won't hit the dance floor until they see someone else making a fool of themselves first. No, they need validation from someone else to take the first plunge, preferably from the risk priests of the startup ecosystem: Accelerators. 


\section*{Accelerators}
\addcontentsline{toc}{section}{Accelerators}
Accelerators do the heavy lifting for angels and other investors; due diligence, market validation, and founder vetting. They also help you turn your raw potential into something that looks more like a business while you are clutching your pitch deck like it's a golden ticket to Willy Wonka's factory of venture capital. When an accelerator stamps your passport, it's like getting blessed by a high lama; suddenly, those risk-tolerant angels start seeing halos where they once saw warning signs.

Unlike Angel's offices, accelerator's gates will keep reopening, giving you multiple chances to perfect your "I'm totally not desperate" face.

\textbf{First attempt:} You walk in confident as a peacock at a bird fashion show, armed with your groundbreaking idea about "Uber for plants" or "blockchain for pet emotions." The partners stare at you with that special blend of pity and confusion, like watching someone try to eat soup with a fork. Rejection email arrives faster than your last overdraft letter.

\textbf{Second attempt:} Three months later, you're back, this time with "AI-powered toasters" or whatever fever dream your sleep-deprived brain conjured up. But wait! You've learned something. Your understanding of the game  is slightly less terrible. The partners' eyebrows raise a microscopic amount higher. Progress! Still rejected, but now with a "Keep trying!" that doesn't sound entirely sarcastic.

\textbf{Third attempt:} By now, you're like Bill Murray in Groundhog Day, except instead of trying to win Andie MacDowell's heart, you're trying to convince seasoned investors that your "social network for introverts" isn't an oxymoron. The beautiful thing? Each rejection must come with a free masterclass in why you suck slightly less than last time.

Here's the universal truth that would make Buddha wiggle: For accelerators, every "no" is actually a "not yet" in disguise. Each application is like a new episode in your personal Silicon Valley sitcom, where the main character (that's you, you gorgeous nerd) keeps coming back with increasingly less delusional ideas. The accelerator partners start recognizing you - not as that stalker who won't take no for an answer, but as that persistent bastard who's actually learning from their face-plants.

\section*{VCs and VC Firms}
\addcontentsline{toc}{section}{VCs and VC Firms}

Oh dear. Venture Capital (VC) is the final boss of the startup funding video game.

\subsection*{A VC firm} The the company VCs work for, Imagine a committee of Gordon Gekkos who’ve traded their Coco for Soylent and their brick phones for iPhones, prowling the Valley in Teslas, hunting for the next unicorn to ride into IPO Valhalla.
\subsection*{The fund} The fund is a number, mostly irrelevant for you, other than to know for how long they have been doing this for, think of it as the number of grey hairs you can count. But basically this is the pile of stash from where their checks come from. Such piles get renewed every time they run out. Named originally: Fund 1, fund 2, fund 3, etc.

\subsection*{VC's}
Bond villain with a Patagonia vest. VC's are the people you interact with and that you partner with. They wield PowerPoint decks like medieval swords, speak in tongues of “runway,” and if you are not careful, they can vaporize your cap table with a single 'let's shake it here'. To a founder, a VC is both the golden ticket and the Wonka who might drop you into the chocolate river for fun. They’ll ask you to “think bigger” while simultaneously reminding you that 99\% of startups are just elaborate bonfires for other people’s money. If Angels are the cool aunts and uncles who slip you \$20 and a wink, VCs are the parents who want to see your report card, meet your friends, and maybe install a camera in your living room. At the pre-seed stage treat them as sophisticated Angels, they will still invest in SAFE's and remember rules \#3 and \#5. Unless ofcourse they say they are interested in a "priced round", then you are in for a Seed stage ride. In which case you MUST read Chapter 3 and 4 (Seed stage important manure you should know).





\section*{Fundraising}

This is no other than convincing people to invest in your company. At pre-seed stage, fundraising is a constant, you are always fundraising and building your business at the same time, so you ARE going to get better at it, every single day, if you spend time to improve from feedback.

\subsection*{Fundraising Process}

Later, this will mean something entirely different, but right now, you just need to do a few simple things: build a wish list of early-stage investors you would like to have on board. This is work that only you can do.

The process will, in part, feel like collecting Pokémon. When it comes to VCs, treat each meeting as a way to get to know each other, and remember that every meeting is an opportunity to ask for an introduction to someone else on your list. You will always need to have your pitch deck with you, and you are going to learn—one way or another—how to pitch.

\subsection*{Pitch Deck}

A pitch is a presentation you give to investors to convince them to invest in your company. This will never be a static document; if your pitch deck doesn't change every time you pitch, you are not learning and you are not improving. Every time you get a "This is interesting," you know what to keep. Every time you get a "No," you need to ask why, and then incorporate those learnings into your next pitch. Many times you will find yourself hand-waving answers to questions; make sure that for the next pitch you have an appendix slide for that. It will show that you are not just really good at making things up on demand, but that you have thought about it all. Your deck needs to have three things: the problem, why this is still a problem, the market (see Chapter 2), and the solution you are building (which must include the validation you have done so far, and this should be different week by week). After that, you simply have appendix slides that keep growing.

\subsection*{Pitching}

Pitching is a great deal of theater, but if you are reading this book, it means you are in for that game. It is the act of giving a presentation to prospective investors and the skill to convince them to invest in your company. This dance at this stage is as close as possible to a one-night stand. Even if you have never participated in one, you can picture yourself in the scenario of what both parties need to agree to such a dance. The decision of early-stage investing happens in such a short time that you really don't know the other person, nor can you tell for sure if this will be great or a disaster. And you have very little information to go on. In fact, as a founder, you have more data than the investors do. So everything you say, how you say it, even how you dress, makes a difference. You dressed like you are going to prom? No, you will exude desperation. You show up without seeing a shower in days? That's a scary place to put money in. You lose your temper when they poke holes in your dreams? Well, if you haven't even kissed yet, what's next if this already a show theyw ant to exit? Anyway, you get the point: each investor interaction is a blank page. Treat it like a blank page and make it count. Perfection your pitch until you see most conversation ending in, what terms are you looking for? how much are you looking for? how soon are you looking for it?

\subsection*{Due Diligence}

Due diligence is a process that investors go through to understand the company and the investment. You are so early that this is really about validating that you are actually building something real, and that you are not just a bunch of raccoons in hoodies. If you are spending too much time on this, something smells like a dead animal, and you are probably doing it wrong.

\subsection*{Closing the Deal}
After each meeting, quickly without delay send them your SAFE agreement and deck, asking for a time for the next meeting. If they are not excited, they will ghost you, and you will move on to the next investor.

\section*{Ghosting and Rejection}
\addcontentsline{toc}{section}{Ghosting and Rejection}

So, the accelerator gods have ghosted you harder than that Tinder date who saw you accidentally snort wasabi thinking it was guacamole, that mythical UC/Stanford professor-turned-investor unicorn is too busy revolutionizing quantum mechanics to care about your startup's distinct lack of citation-worthy breakthroughs (which is totally fine by the way - remember Airbnb started by renting air mattresses to broke conference attendees, not exactly Nature publication material). Your idea might not be splitting atoms or bending spacetime, but hey, often-times the path to billions is paved with simple solutions to everyday problems. Welcome to Plan C, mijo - the path of pure, unfiltered hustle through the digital trenches of LinkedIn and other networking shenanigans.
Remember Rule #2 like it's your new religion: Never. Ever. Give. Up. Your startup might be held together with eye booggers and dreams right now, but that's no reason to throw in the towel. Time to channel or give birth to your inner digital stalker (the professional kind, not the restraining order kind). Because here's the beautiful truth about the Bay Area: it's a graveyard of failed startups, and every tombstone marks someone who might just help you avoid their fate. These beautiful disasters, these glorious failures - they're your new best friends. They've been through the meat grinder and came out the other side with two priceless things: battle scars and contact lists.
Fire up LinkedIn like it's your personal dating app. But be clever about it, you magnificent spam artist. Don't just blast "PLZ INVEST" messages like a Nigerian prince with a startup fetish. Start with the walking wounded - those who've recently tasted the bitter wine of startup mortality. Slide into their DMs with the grace of a caffeinated ninja:
"Hey [insert name], saw you built [their failed startup] - fascinating approach to [whatever they tried to fix]. Working on something similar, would love your battle-tested perspective. Coffee?"
Some will ignore you harder than a cat ignores its expensive bed. Others will respond with war stories and wisdom. And here's where the magic happens: never leave a meeting, even a "no," without asking The Question™: "Know anyone else who might enjoy watching me try to defy gravity with this particular brand of sexy?
Remember: keep pushing, keep hustling, keep spamming (tastefully, like a Shamanic goddess trying to sell enlightenment on Instagram - just enough wisdom to intrigue but not enough to get banned from the algorithm). Eventually, like watching a Buddhist monk discover energy drinks for the first time, someone will look at your duct-taped dreams - held together with the determination of a squirrel building its last nest before winter - and think, "You know what? This particular flavor of crazy might just work, as discovering vodka pairs surprisingly well with pickle juice at 3am." And that's when the real fun begins. Your first SAFEs!

