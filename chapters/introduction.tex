\chapter*{Introduction}
\markboth{INTRODUCTION}{INTRODUCTION}
\addcontentsline{toc}{chapter}{Introduction}

If you have it in you. Whatever that is, there's something seductive about the Bay Area, something that draws you in like a moth to a flickering light. But behind that glittering surface, there's a dark swirl of uncertainty, where even the boldest ideas can live in constant superposition of death and alive. That is what this book is about. Not the glossy headlines or the Silicon Valley mythology, but the real pains: the blood, the sweat, and the game of hidden rules I wish we had known about before diving headfirst into the madness of being a founder looking for funding.

\subsection*{Why this book?}
Let me briefly introduce myself and explain why I wrote this book. I'm JT, CEO of MindsDB, a company I co-founded with my dear friend Adam C. In the end of 2023, we raised a \$47M seed round from some of the best investors in the world. How did we, the chihuahuas in a pitbull fight, managed to do that?—that's the question I get from many early-stage founders or aspiring founders. Now, depending on the circles you run in, \$47M might be seen as either a drop in the ocean or a staggering amount of cash. But, in most circles, that's a pretty decent seed round.

For us, it didn't come easy, and it sure as hell didn't happen overnight. Before that seed round, we had gone through a wild journey of raising about \$8 million in pre-seed money, and so much could have been done differently. So, a year ago, as a way to blow off steam, I decided to write 3-4 pages every weekend to send to those who ever so often come to me and asking fundraising questions and like us—plucky little scamps—are walking that early startup journey and need funding to make it happen.

What matters here isn't just how much we raised, but the brutal learning curve it took to get there—the fuckups, the close calls, the lessons they don't teach you in any classroom or startup accelerator. And trust me, we've had our share of those. There are great books on the business principles needed to get from nothing to "scale up" (and please, read them). These tales are only from the fundraising perspective. We learned that no one gives you the real picture for surviving the actual sh*t and the unspoken rules that can make or break your first big round. Almost no-one tells you how it feels to walk out of a dead-end meeting with a prospective investor, licking the bruises and gashes of a pitch that went nowhere, questioning if you're pushing forward with genius or just a dumb idea that's about to implode.

\subsection*{Who is this book for?}

Specifically, this book is for you if you're somewhere between "I have an idea, this idea needs some moolah to get it off the ground" and "I am just working on a seed round that would make grandma faint." The focus here is just the fundraising part, what is covered in this book is just but one part of a long journey of building a company.

This book is for the dreamers who've decided that working for someone else's dream is about as riveting as watching someone else's loading bar reach 99\%. In particular, I wrote this for the underdogs. The ones who don't have a Stanford MBA, a family trust fund, or a cousin who's best friends with Marc Andreessen's barber. 

You? you could be anyone, but, If I could take a wild guess you are one of the following:

\subsubsection*{You're are fresh out of college or before that}
Congratulations, you still get carded at the bar and your liver is younger than most houseplants.
Still clueless after graduation? Welcome to the club of many! If you have no idea what to do with your life, join a startup that’s tackling a problem so big it makes your parents question your sanity. Worst case, you’ll learn what not to do. Best case, you’ll accidentally stumble into your life’s work while eating ramen at 2am with people who also have no idea what they’re doing. Either way, it beats living in your mom’s basement explaining to her why your crypto portfolio is “just consolidating.”

You are that person that has found a problem, has prototyped a solution during your free time and no matter what you do, for as ugly as your solution is, the problem is so insane that people keep asking you for it? Well, you are in the right place, but you will not need this book for too long. You are as rare as a monkey riding a Segway through a Taco Bell drive-thru at 3am, investors will chase you. Enjoy your brief stay here, you beautiful anomaly.

\subsubsection*{You're still working that cushy corporate gig}
There's a greater than zero chance that you're still working that cushy corporate gig, the one with the nice benefits, the reliable paycheck, the reassuring stability of a 401k that seems to stretch out into your comfortable retirement. You're good at navigating the slow, bureaucratic machinery of the corporate world, learning to squeeze the teats of some incredible product-market fit that someone else created years ago. And let's face it—you've gotten pretty damn good at milking it, haven't you? But every once in a while, something inside you stirs. Something tells you that maybe—just maybe—you're wasting your life. Maybe you're like I was, sitting in those endless meetings thinking, "What would happen if I just said, 'fuck it' and did my own thing?"

If you and I are made of the same, that itch doesn't go away. I believe, It is very likely that one day, too old to hold your farts, while reflecting on the reflection of your saggy butt cheeks in the mirror, you'll look back at your safe life, your stack of savings, your perfectly planned but shrinking future, and you'll ask yourself between the tortures of I wonder what woulds: What would have happened if I asked for that phone number,... what might have been if I'd just taken the leap,... what did I miss?. 

You have mouths to feed? buckle up honey, you are in for a fucking scary ride.

\subsubsection*{You're already a founder}
Maybe, naive enough, you've already made that leap, and… you're here because you're drowning or like a confused pinguin desperatly flapping its wings trying to fly before it hits the ground. Trust me, I know how it feels. One day, you're walking on water, feeling like you've cracked the code, like the world is suddenly obvious. And the next, you're gasping for air, trying to claw your way out from under an iceberg that's crushing you. Welcome to the rodeo cowgirls and cowboys. You've chosen this life, and hopefully this book helps you f'it up a bit less.


\subsection*{So, what can you expect?}

I hope it is abvious to you, that I didn't write this book to give you a blueprint for success, because I truly don't know one. But I do know that success in this place can feel as arbitrary as the wind, a chaotic, bipolar storm that can either propel you to heights you never imagined or sweep you into the gutter. 

This book is a response to all the things we didn't know as early stage founders, all the fundraising lessons we had to learn the hard way. If I could go back and do it all over again, I'd change so much that we'd have a few shaved years off the grind, or at least of barking at the wrong trees. This book is about the framework we built after falling right on our faces enough times to recognize a pattern. It's a framework for seeing the true game, not just the surface-level mechanics of pitching. If you're here expecting some feel-good advice or a shortcut, you're in the wrong place. Like a startup, this is about raw survival, about making fewer mistakes and cutting through the bullshit that drowns 90\% of startups in this small town. And let me make one thing clear: this book is for you only if you're dead serious about getting your butt to the Bay Area and do your fundraising here (Maybe, maybe New York, but definitely not somewhere else). If you're dreaming of making it big from the comfort of your hometown or thinking that Zoom calls are going to cut it—you're reading the wrong paper. The Bay Area is its own animal, and people dance differently here. You either play by some unwritten rules that every veteran seems to know or you get eaten alive (you'll probably eat yourself alive, we will talk about that later).

We made so many mistakes, who knows, maybe it's a miracle we survived. But each fuckup became a turning point, a chance to course-correct, and eventually, we started to see the underlying patterns in the chaos. These moments weren't just about improving our pitch. They were about stripping away the bullshit and asking ourselves the hard questions: Why the hell are we doing this? What's the real problem we're solving? In 5-10 years, will I still want to be working on this stuff? Why should anyone (me included) care? And most importantly, Are we even working on a problem big enough for this to be worth it?

We realized something most founders miss: it's not just about pitching better. It's about fixing the core fundamentals of your business, so the pitch takes care of itself. Hopefully you will see that when you solve the right problems—real problems—people want to throw money at you. And I know how crazy that sounds, because at the start, we didn't believe it either. But trust me: Once you hit that sweet spot, once you align everything just right, investors will chase you. There's nothing more surreal and mind-fucking than saying "no" to someone's money, not because you don't need it, but because someone better for your company just walked into the room.

But before you get there, and hell! for sure after you get there, you're going to fuck up. A lot. And that's okay. Because every mistake you make brings you closer to seeing what's really going on beneath the surface. Every rejection, every failed pitch, every "don't call us, we'll call you" moment—those are the moments that teach you how to really play this game. Because deep down: IT IS A GAME. JUST PLEASE DON'T MAKE THE SAME MISTAKES TWICE.

So, here we are. You've read this far, which means you're either serious in wanting to be an entrepreneur, an early stage founder on the road to your first round. Or you're at least curious enough to stick around, and god forbid tomorrow is the day you tell your boss that you hope the message finds them well, but this is the day you quit. Your keyboard is still warm from a one way ticket purchase to SFO. Good. This book is your guide, not a formula, but your rough roadmap to navigating the absurd, unpredictable, and utterly fascinating world of massive seed rounds in the Bay Area. And if we're lucky, it'll save you from making the same mistakes we did. But remember—this is just a guide. The real journey is yours, and no one can walk it for you.

Let's dive in.
