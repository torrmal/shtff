\chapter{Getting Pre-Seed Money Tutorial}

You've started on your tighty whities, standing at the edge of a financial cliff, wearing nothing but your dreams and a pitch deck. The wind of market uncertainty is whipping through your metaphorical hair, and somewhere in the distance, an investor is probably saying "interesting" while mentally planning their dinner. But how do you get there? how do you convince them investors to join your delusion?

\begin{quote}
    ``No rain, no rainbow 🦄'' \\
    \hfill --- Philosopher Joanna Slupczewska
\end{quote}

The short answer? It will be a bumpy ride, full of times when you would like to quit, but don't. Welcome to this so-called "tutorial"—although, let me be clear: unless you get out there and personally get kicked on the face, this tutorial is as useful as watching heart surgery tutorials on YouTube at 2 AM. Now that you understand how investors think (risk), and also the lingo in this town and the rules of this game. You're about to learn how to raise pre-seed money, which is essentially speed dating, but instead of finding someone who'll tolerate your weird laugh, you're finding people who'll tolerate your weird business model.


Obvious or not, here are the things to understand:


\section{PRE-SEED IT IS NOT a Single Round; It's an Ongoing Round}


Here's where the illusion shatters, but trust me, it is actually a good thing: raising at pre-seed isn't a dramatic all-or-nothing moment with champagne and applause. It's an ongoing battle, a financial siege. Investments at this stage are a mosaic of SAFEs and convertible notes, pieced together as you scramble for validation and resources.  I am sorry to break to you, but you probably won't get one big, gleaming check; you'll get trickles, possibly starting with breadcrumbs that keep you alive while you scrape for the next. It's not linear; it's an unpredictable patchwork. And here's the paradox: as your idea becomes less terrible, your terms get better. What starts with a few tens of thousands can swell into millions if you prove that you're more than a flash in the pan.

The beautiful absurdity of it all: in the beginning, everyone's waiting for someone else to make the first move. It's like trying to start a conga line at a particularly stiff corporate party: nobody wants to be the first to wiggle their hips. But once a few people join in, suddenly everyone's an enthusiastic dancer.

The psychology works like this: Walking into a room announcing "I'm raising a few million" with empty pockets is like trying to sell timeshares on Mars. But watch how the story transforms: "We're raising 100k and already have 50k", now you're selling tickets to dudes into a party that's already buzzing with chicks on a bachelorette party from Sweden. "Raising 300k with 100k in the bank", the intrigue builds. "Looking for 1M with 300k-600k locked in", if this happens quick enough, suddenly you're the hottest ticket in Silicon Valley.

Now, as we dive in details, here comes the delicate art of terms (keep them YC-safe compatible, as we discussed in the startup lingo chapter). The "cap" (look it up in the lingo chaper too, but a refresher): your theoretical valuation, is like a price tag in a parallel universe where crypto only goes up. It's a number that exists in the realm of "trust me, bro" until enough people nod sagely and say "seems legit."

Start with a cap that makes early investors feel like they're getting in on the ground floor of the next big thing. Then, as more checks materialize, gradually adjust that number upward, like a master chef adding salt; carefully, tastefully, but with conviction. Check Crunchbase or make a friend that will let you peek at their Pitchbook to see what other visionaries in your space are commanding. It's like a sophisticated game of pricing psychology where everyone pretends the numbers are based on complex mathematical models, when really it's more like interpretive dance called the pre-seed evolution!, where your funding journey transforms in stages like a Pokemon with anxiety issues. 

\subsection*{Stage 1: The "Please, Anyone?" Phase (\$10k-100k):}

This is where you're basically selling hope wrapped in a PowerPoint. Angels look at you the way a parent looks at a child trying to explain why drawing on the walls is actually not good for property values. But here's the thing: you only need one believer, one beautiful optimist who sees past the duct tape and dreams to your actual potential.

\subsection*{Stage 2: The "Hey, We're Kind of Real" Phase (\$100k-200k):}

Suddenly, you have that magical first check. It's like getting your first kiss; awkward, probably could have gone better, but hey, someone thought you were worth it! Now you can walk into rooms saying, "We've already raised..." which is startup speak for "See? Someone else thinks we're not completely crazy!"

\subsection*{Stage 3: The "Now We're Cooking" Phase (\$200k-500k):}

This is where the momentum builds faster than technical debt in a hackathon. Each "yes" breeds more "yeses" like rabbits discovering Viagra. Investors start responding to your emails faster than they ghosted you in Stage 1. You're no longer pitching; you're "updating interested parties on your round's progress." Fancy!

\subsection*{Stage 4: The "Is This Real Life?" Phase (\$500k+):}

Welcome to the line to the tickets for the entry leagues. You are raising \$1m and you already have \$500k, And now you know how this game works, you can go up to a few million this way. You can choose to dance this conga to your own playlist, but here once again, a lot of this dance is accelerated if you join an accelerator, and if you do, of course, please do it right, because there is a wrong way for sure.



\section{Time Is Money— And You're Bleeding Both At The Same Time}

Now that you know how this dance will evolve into a wallet striptease, let's cover the other variable, time. I guess Time and Money are what everything boils down to, in the future you can use one to aquitre the other, however in early pre-seed it's a bit different as you don't have much of either.

\subsection*{Have you seen the movie "He is just not that into you"?}

Learning how not to waste time is a life or death situation here. I now know that the brass knuckles of our journey weren't just about our stubborn refusal to give up (also important). However, more importantly, it was about letting go of potential investors that were a waste of time, isntead we made the mistake of confusing perseverance with intelligence, and it almost killed us; We became the startup equivalent of that person who thinks is your ex, just because you exchanged a few flirty lines, and since then, just won't stop texting! Sadly, early on we haunted investors until they came back with messages like: "Look, I was planning to ghost you, but before I file a restraining order, let me be crystal clear: It's not a maybe, it's a hard no. Not now, not ever, not in this universe or any parallel dimension where unicorns tap dance on Mars to wallabies playing in a band. I've got a million other deals I haven't met yet, and your constant messaging is making my phone vibrate off my damn desk."

So what did we learn? Well, we learned something so fundamental it should be taught in preschool right after "How to Sign Your Name on Crayon Masterpieces 101." Picture this conversation:

\begin{quote}
"What did you learn today, gummybear?"

"I drew this monkey, Daddy! And I learned that you shouldn't waste a single precious millisecond extra with early-stage investors that are not into you!"

"That's my little girl! Your life expectancy just increased by 40\% if you ever decide to be one of them interpernurs..."
\end{quote}

Mastering the art of moving on should come as naturally as ghosting comes to investors. Here’s the golden protocol: if you spot those classic signs of ghosting—let’s say you’ve fired off two emails and received nothing but the cold embrace of silence—just move on. Early-stage investors are especially allergic to saying “no” because, deep down, they fear you might be that one-in-a-million unicorn and they’ll spend eternity kicking themselves for missing it. They all know rule #1: All startup ideas are shitty ideas until they aren't. So, Don’t worry; if your round starts to catch fire, you’ll be amazed how these same investors will reappear from the shadows with, “Hey sexy! Could you send your deck again? My dog ate the last one… and also my memory.”

But beware: there is a monster far deadlier to your time (and sanity) than the vanishing investor: we are talking about the investor who hangs around just to fuck around, dangling hope like a laser pointer in front of a sleep-deprived cat. At least a ghost leaves you in peace. This one will have you chasing your tail until you forget what a life not spent checking your inbox felt like. Lucky for you, There is a pattern and a uno card for this one too:

\subsection*{Pre-seed due diligence - be aware of this pattern}
I often think of those very early times when investors asked for due diligence? were they interested? well we were on idea stage (you probably are too, and that's okay), the point is, we were more early stage than a caterpillar's first identity crisis! That "due diligence", was just them using us as their personal market research interns. They knew it was all speculation at this point; like trying to predict which way a drunk penguin will waddle. So if you are going to indulge in due diligence, at least do yourself a favor: learn, learn something new every time you are asked to jump through their hoops like circus seals chasing premium fish. 

So what's the protocol when investors start asking for your grandmother's DNA and a detailed analysis and all things outside of your deck and investor room folder? Here's the deal: ghosting is like a dance-off at a startup mixer; it takes two to tango, baby! But trust me, your response should NOT depend entirely on your dance card.  If you are the belle of the ball with VCs throwing term sheets at you like confetti, or are you the awkward wallflower who just got their first "maybe" after 47 rejections?

Either way, channel your inner time management guru and remember: your time and their time is worth its weight in Bitcoin (circa whenever it's high as Friday night Snoopdog). Sure, if you want their money, you can play along and say "Absolutely, I'll send you whatever documentation your heart desires!" But here's the twist: \textbf{make it conditional on a SAFE agreement}. The, show me yours and I will show you mine paradox. That means they commit to investing by giving you a SAFE, if they like what they see, then they wire the money, and on the other side you commit to taking their money if their due diligence doesn't reveal that you're actually three raccoons in a trench coat. It's way better than becoming their unpaid market research intern while they dangle funding in front of an increasingly desperate you.



\section{Avoid the Wrong Investors}

The earlier you are, the less obvious this gets. Once your fundraising game has some mileage, you’ll mostly be courting VCs, the kind of people whose names you can Google to find out if they’re legit or not. But in those squishy, larval days of your startup life, you’re splashing around in the kiddie pool with angels, randos, and anyone who so much as replies to your LinkedIn message at 3 a.m. In this slippery stage, spotting the wrong investors can feel harder than finding a vegan at a Texas BBQ. So let me drop a that will make it very simple to dodge the most common mistakes when it comes to “early financial backers”, hear me out and repeat:

 \textbf{``Only take money from people who can laugh while watching it evaporate.''} Not because you're planning to fail ---you'll work harder than a rat on a cocaine-coated wheel--- but because this is pre-seed, baby. The odds against you are about as friendly as an Australian Magpie in mating season. Pre-seed is not a garden; it’s a battlefield. It’s where you sharpen your teeth on the grindstone of doubt, where your idea sheds its soft, useless parts and becomes something sharp enough to cut in real interest. So make sure your backers truly understand what level of risk they are betting their money into.

\subsubsection*{Warning: Friends, fools, and family round}
Some call pre-seed that, and if you ever hear this advice, heed me: \textbf{run}. Run like hell. You are not a con artist peddling snake oil under the guise of a brilliant idea!!! \textit{Nein!} As just explained, and worth diving deeper: The pre-seed stage is the riskiest chapter in a startup’s life cycle, a place where even the most determined founder is more likely to fail than to thrive. This is where the unvarnished truth comes out: (once again, repeat after me, 5 times) \textbf{``DO NOT take the money that grandma saved for your mother’s rainy day!''} At this stage, when you can’t tell your ass from your hand, that money is better off in the S\&P 500 or parked in some safe, diversified corner of the financial universe.

The sad reality is, believe it or not, and god bless them, there’s no shortage of good souls, eager to believe and eager to part with cash they can’t afford to lose. It might seem tempting when you're desperate to make your idea fly, but taking someone’s life savings, money they bled years to accumulate, is wrong, and unless you are an asshole, it is the fast track to sleepless nights, moral bankruptcy, and the guilt that will haunt you long after the startup burns out. The path is lined with those willing to bet everything, from the trust fund artist looking for meaning to the retiree with savings that shouldn’t be gambled on a pipe dream. These are the ones you must resist. In the pre-seed stage, don’t take money that isn’t designed to risk disappearing. It’s not just about doing the right thing---it’s about ensuring that the support around you is rooted in smart, informed decisions. At this stage you cannot be here to be someone’s miracle or savior; you’re here to build, to prove, and to play the long game with the right backers who understand exactly what they’re in for.

\subsubsection*{Warning: You'll meet the "advisors"} 
Those beings who offer their "expertise" in exchange for equity. They'll promise introductions that never materialize, like crypto gains in a bear market. When they come at you with their advisory agreement PDFs, hit them with the ultimate uno reverse card: "Love your enthusiasm! How about you put some skin in the game? A small check would really align our interests..." Watch them disappear faster than free pizza at a developer meetup or you will find what you want which is that first round of checks.

\subsubsection*{From whom then?}
Acknowledge that for the true investor, the pre-seed stage is a numbers game, and it demands a certain kind of participant---two, in fact: the informed and the strategic. They both spread their bets thinly across many hundreds of startups, knowing that most will fail but one or two will soar high enough to cover the losses many times over.

\section{So, what is the difference between a pre-seed stage and a seed stage?}

We will cover Seed stage and Seed rounds in later chapters as well as in the Lingo Chapter. But understand that black and whites are not a thing in early stage. For you, you are in a journey of building a company, this language, pre-seed, seed, series A, etc, its fundraising lingo, and this language helps you set the expextations for when you are asking for money.

\subsubsection{I am in PRE-SEED stage?} 

When you tell investors that you are in the pre-seed stage, the good investors (meaning those who are used to investing in early-stage startups) will know that they are making investments based on how promising the opportunity looks just by examining the problem you are trying to solve. They will decide whether to invest based on the two risks we cover in Chapter 2: Market risk and Founder risk. The checks you receive will not be given at a set valuation, but rather with the prospect of a future valuation. This means that, at this stage, no one can really say how good the opportunity is, because you are still selling a vision rather than a proven business. They are literally making a bet—please read the glossary for details about convertible notes and SAFE agreements. Essentially, your value is unknown; you don’t even have a product yet, but you are convincing people that the problem you are setting out to solve is worth investing in. If you succeed, they hope to turn a relatively small check (tens of thousands to a few million dollars) into a significant return. Investors need to be able to imagine you as someone who could build a wildly successful business (that means picturing you with one to ten horns on your forehead and shitting rainbows). 

So, if you are talking to an investor whose LinkedIn profile says “later stage, (A,B,\,$\Omega$)” you are barking up the wrong chickens. The best thing you can do is ask them for an introduction to someone who invests at the early stage. That introduction will help you get connected in the right direction.

\textbf{What is pre-seed money good for, you may ask?} Well, it's to pay the team, to cover operational costs, and ideally to get you to a point where you can focus solely on building and validating with customers, users that it's not just you and yoru grandnan that think this is a good idea. So eventually, you will be able to have 18–36 months or so of runway ahead of you without having to spend all your time fundraising is, for the most part, that is what a SEED round will likely provide.

\subsubsection{So, am I in SEED stage?}

Well, this is something that happens a bit more organically than you might think.. you stumble into seed stage. The moment someone wires you money, you’re “investor-backed”—which sounds glamorous until you realize it’s just you, a spreadsheet, and a redbull can at 2 a.m. The important thing to know is that as you collect them signatures on SAFEs, you must, in tandem, like a juggler of shit-on-fire, you shall perform the ancient art of “progress.” This means transforming your idea from “wouldn’t it be cool if…” to a “look, it actually solves something!”. This metamorphosis is not overnight. Your business will slowly ooze from “pure fantasy” to “mildly convincing hallucination”.  Suddenly, you’re not just selling air in a can—you’re selling oxigen for oxigen deprived, with a label and a barcode. That’s when early-stage investors start muttering, “Maybe it’s time for a priced round,” like wizards debating whether it is time to summon a demon or to give you sparklers until you grow a pair of teeth. We’ll unravel that particular demon spell in the following chapters, and in the glossary, because you’ll need a talisman.

Anyway, If you’re not hearing these magic words: "Fuck it, how about we do a priced round?", yet. Focus on building your business, and—just as importantly—on protecting your own and your company’s collective butts from the two cosmic banana peels we discuss in Chapter 2 (market and founder risk). Until then, set expectations low and keep telling people you’re in the pre-seed stage (Even if you think you are seed stage).





